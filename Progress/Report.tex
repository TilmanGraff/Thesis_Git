\documentclass[11pt, oneside]{article}   	%standardeinstellung
\usepackage{amsmath}				%damit du formeln eingeben kannst
\usepackage[parfill]{parskip}    			% neuer absatz durch leere zeile
\usepackage{graphicx}				% bilder einfügen
\usepackage[onehalfspacing]{setspace}	%1,5 facher zeilenabstand
\usepackage{hyperref}				%damit man links erstellen kann
\usepackage{float}					%intelligentes anpassen von bildern, damit es keine lücken im text gibt
\graphicspath{ {figures/} }				%automatisches erstellen von "List of Figures"
\usepackage[font=small,labelfont=bf]{caption} %formatierung der bildunterschrift
 \usepackage{geometry}				%Seitenlayout
 \usepackage{tikz}
 \usetikzlibrary{shapes,arrows}

 \geometry{						%musst du nicht unbedingt selbst einstellen
 a4paper,
 total={210mm,297mm},
 left=15mm,
 right=15mm,
 top=25mm,
 bottom=20mm,
 }

\usepackage{amssymb}
\usepackage{caption}
\usepackage{natbib}
\usepackage[utf8]{inputenc}
\usepackage{amsmath}
\usepackage{amsfonts}
\usepackage{amssymb}
\usepackage{wrapfig}
\usepackage{subcaption}
\usepackage{pdflscape}

\def\checkmark{\tikz\fill[scale=0.4](0,.35) -- (.25,0) -- (1,.7) -- (.25,.15) -- cycle;}

\title{Progress Report for MPhil Thesis}
\author{Tilman Graff -- University of Oxford}
\date{\today}

\begin{document}

\tikzstyle{data} = [draw, ellipse, fill=blue!20,
    text width=4.5em, text centered, node distance=3cm, minimum height=4em]
\tikzstyle{input} = [draw, ellipse, fill=red!20,
    text width=4.5em, text centered, node distance=3cm, minimum height=4em]
\tikzstyle{output} = [draw, ellipse, fill=green!20,
    text width=4.5em, text centered, node distance=3cm, minimum height=4em]
\tikzstyle{program} = [rectangle, draw, fill=white!20,
    text width=5em, text centered, rounded corners, minimum height=4em]
\tikzstyle{line} = [draw, -latex']

\maketitle

\section*{Research Question}
\textbf{Which factors influence the global distribution of trade network optimality?}

In this thesis, I aim to create a (potentially) global dataset of trade network efficiency. Taking the spatial distribution of current economic activity and population as given, I use a model from a recent working paper to determine the optimal trade network for each country. I then compare each country's current road network to its optimal one and derive a measure of how far a country is currently away from its ideal self.

In a second step, I will then investigate the origins of this global distribution. Which factors led to the heterogeneities among countries today? Specifically, I will look at:

\begin{itemize}
  \item Do networks with large colonial infrastructure investments do better or worse today?
  \item Does tribal favoritism explain why some areas are lacking lucrative investment?
\end{itemize}

\section*{Process}

In order to conduct this research, I will need to follow a series of steps and transfer data between multiple programming softwares. Here is a step-by-step guide:

\begin{itemize}
  \item[\checkmark] Find global raster data on population, night-lights, ruggedness, and colonial infrastructure investments.
  \item Grid the world on 50x50km squares and aggregate finer-resolution data into those grids.
  \item Locate the maximum population point within each grid and call this the (population) centroid of the grid.
  \item[\checkmark] Use OpenStreetMap to find distance and average speed between neighboring gridcells.
  \item Use distance and ruggedness to calculate Infrastructure Building Cost Matrix $\delta^{I}_{i,k}$ for every country.
  \item Use average speed to calculate current Infrastructure Matrix $I_{i,k}$ for every country.
  \item Use distance to calculate (iceberg) Trade Cost Matrix $\delta^{\tau}_{i,k}$ for every country.
  \item Use $\delta^{I}_{i,k}$, $\delta^{\tau}_{i,k}$, and $I_{i,k}$ to find the optimal trade network $I^{*}_{i,k}$ and optimal tradeflows $Q^{*}_{i,k}$ for every country. This directly follows the Schaal Working Paper. I know how to do this, I am however afraid that this might take too long.
  \item Compare $I_{i,k}$ and $I^{*}_{i,k}$ for every country to obtain a measure of network optimality $\zeta_{c}$ for every country $c$.
  \item Investigate heterogeneity in $\zeta_{c}$

\end{itemize}

This flowchart visualises the process. It shows the path input data (red circles) take through various programming languages (white rectangles) and intermediate datasets (blue circles) into eventual findings (green circle).

\resizebox{1\textwidth}{!}{
\begin{centering}
\begin{tikzpicture}[node distance = 3cm, auto]
  \node [input] (Lights) {Night Lights};
  \node [input, below of=Lights] (Rugg) {Ruggedness};
  \node [input, above of=Lights] (Population) {Population};
  \node [program, right of=Lights] (QGIS) {QGIS};
  \node [data, right of=QGIS] (Centroids) {Grid Centroids};
  \node [program, right of=Centroids] (R) {R};
  \node [input, above of=Centroids] (Roads) {Road Network};
  \node [data, right of=R] (Infrastructure) {Infrastr. Matrix};
  \node [data, above of=Infrastructure] (Trade) {Trade Cost};
  \node [data, below of=Infrastructure] (Building) {Building Cost};
  \node [program, right of=Infrastructure] (Matlab) {Matlab};
  \node [data, right of=Matlab] (Opt) {Optimal Network};
  \node [program, below of=Building, node distance=5cm] (Stata) {Stata};
  \node [input, right of=Stata, node distance=4.5cm] (misc) {Expl. variables};
  \node [output, left of=Stata, node distance=4.5cm] (Findings) {Findings};

  \path [line] (Lights) -- (QGIS);
  \path [line] (Population) -- (QGIS);
  \path [line] (Rugg) -- (QGIS);
  \path [line] (QGIS) -- (Centroids);
  \path [line] (Centroids) -- (R);
  \path [line] (Roads) -- (R);
  \path [line] (R) -- (Infrastructure);
  \path [line] (R) -- (Trade);
  \path [line] (R) -- (Building);
  \path [line] (Building) -- (Matlab);
  \path [line] (Infrastructure) -- (Matlab);
  \path [line] (Trade) -- (Matlab);
  \path [line] (Matlab) -- (Opt);
  \path [line] (Opt) -- (Stata);
  \path [line] (Centroids) -- (Stata);
  \path [line] (misc) -- (Stata);
  \path [line] (Stata) -- (Findings);

\end{tikzpicture}
\end{centering}
}


\end{document}
