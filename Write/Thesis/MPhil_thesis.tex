% Preamble
\documentclass[11pt, oneside]{article}   	%standardeinstellung
\usepackage{amsmath}				%damit du formeln eingeben kannst
\usepackage[parfill]{parskip}    			% neuer absatz durch leere zeile
\usepackage{graphicx}				% bilder einfügen
\usepackage[onehalfspacing]{setspace}	%1,5 facher zeilenabstand
\usepackage{hyperref}				%damit man links erstellen kann
\usepackage{float}					%intelligentes anpassen von bildern, damit es keine lücken im text gibt
\graphicspath{ {figures/} }				%automatisches erstellen von "List of Figures"
\usepackage[font=small,labelfont=bf]{caption} %formatierung der bildunterschrift
 \usepackage{geometry}				%Seitenlayout
 \usepackage{tikz}
 \usepackage{lipsum}
 \usetikzlibrary{shapes,arrows}

 \geometry{						%musst du nicht unbedingt selbst einstellen
 a4paper,
 total={210mm,297mm},
 left=15mm,
 right=15mm,
 top=20mm,
 bottom=20mm,
 }

\usepackage{ragged2e}
\usepackage[export]{adjustbox}
\usepackage{dsfont}
\usepackage{amssymb}
\usepackage{caption}
\usepackage{natbib}
\usepackage[utf8]{inputenc}
\usepackage{amsmath}
\usepackage{amsfonts}
\usepackage{amssymb}
\usepackage{wrapfig}
\usepackage{subcaption}
\usepackage{pdflscape}
\usepackage{filecontents,url}
\usepackage{soul}
\usepackage[space]{grffile}
\usepackage[UKenglish]{isodate}
\usepackage{anyfontsize}
\def\one{\mbox{1\hspace{-4.25pt}\fontsize{12}{14.4}\selectfont\textrm{1}}} % 11pt

\let\oldref\ref
\renewcommand{\ref}[1]{(\oldref{#1})}

\newcommand{\mysubcaption}[1]{
\justify
\begin{spacing}{0.7}
\textit{\footnotesize #1}
\end{spacing}}

% \title{Spatial Inefficiencies in Africa's Trade Network}
% \author{Tilman Graff -- University of Oxford}
% \date{\today}

\begin{document}

\bibliographystyle{cantoni_copy}
%\maketitle
\begin{titlepage}
    \begin{center}
        \vspace*{0cm}

        \Huge
        \textbf{Spatial Inefficiencies in \\
        Africa's Trade Network}

        \vspace{0.5cm}


        \vspace{1.5cm}
        \LARGE
        \textbf{Tilman Graff}

        \vfill

        \includegraphics[width=0.4\textwidth]{/Users/Tilmanski/Documents/UNI/MPhil/Second Year/Thesis_Git/Write/Thesis/beltcrest.pdf}

        \vfill

        Submitted in partial fulfilment of \\
        the requirements for the degree of the \\
        Master of Philosophy in Economics

        \vspace{0.8cm}



        \large
        St. Antony's College\\
        Word Count: 123 Words\\
        \today

    \end{center}
\end{titlepage}

\newpage

\begin{abstract}
  Are African roads where they should be? I assess the efficiency of transport networks for every country in Africa. Using rich data from satellites and online routing services, I simulate optimal trade flows through more than 90,000 links for the entire continent. Employing a recently established framework from the optimal transport literature, I maximise over the space of networks to find the optimal road system for every country. Where would the social planner ideally build more roads and which roads are superfluous in promoting beneficial trade? Comparing current and optimal network, I then construct a novel dataset of local network inefficiency for more than 10,000 African regions. I analyse roots of the substantial variation present in this measure. I find that colonial infrastructure projects from more than a century ago still persist in significantly skewing trade networks towards a sub-optimal state. Areas close to colonial railroads are about 1.7\% too well off given their position in the network.
\end{abstract}

% End Preamble

%-------------------------------------------------

\section{Introduction}

\section{A model of Optimal Transport Networks}
% Chapter
Are African roads where they should be? To gain a notion of transport network efficiency, the first step is to identify the optimal allocation of roads for a given country. If a social planner were to observe a country's geography, where would she place highways and footpaths, and which regions would she leave unconnected? Intuitively, this optimal network should succeed in connecting regions with a heavy interest in beneficial trade, place roads only where they are cheap to build, and avoid overspending on roads that nobody ever uses.

Optimising over the space of possible network has proven to be challenging to researchers and policymakers alike. The many contingencies and non-linearities of the network structure make it hard to obtain a unique closed-form solution to the problem. The large number of possible connections between locations have, furthermore, led to computational challenges. Researchers and practitioners have hence often been left only with slow, iterative algorithms to obtain a notion of local network optimality. A recent contribution by \cite{fajgelbaum_optimal_2017}, however, manages to circumvent these issues. Under a number of reasonable assumptions, it paves the way for obtaining closed-form solutions to the problem of optimal transport networks without straining computational capabilities all too much. It thereby allows for nesting many of the standard models of the trade literature and can thus flexibly be tailored to the question at hand. This thesis harnesses a version of the \citeauthor{fajgelbaum_optimal_2017} framework to compute the optimal transport network for every African country. The model is introduced in the following paragraphs, chapter \eqref{chapter:calibration} then describes the procedure used to bring the model to the data.

\subsection{Geography}

Following the set-up and notation of \cite{fajgelbaum_optimal_2017}, I consider a set of locations $\mathcal{I} =\{ 1,...,I\}$. Each location $i \in \mathcal{I}$ inhabits a number of homogeneous consumers $L_{i}$. This number is treated as given and fixed for every location, such that consumers are not allowed to move between locations. Each consumer has an identical set of preferences characterised by
\begin{equation*}
  u = c^{\alpha}
\end{equation*}
where $c$ denotes per capita consumption. Due to the symmetry of consumers in every location, per capita consumption (and therefore utility) is identical for every consumer within each location, such that every consumer in location $i$ consumes $c_{i}$. It is sensible to further denote
\begin{equation*}
  C_{i} = L_{i}c_{i}
\end{equation*}
as total consumption in location $i$.

There there is a set of goods $\mathcal{N}$ denoted by $n =\{ 1,...,N\}$. Total consumption in each location is defined as the CES aggregation of these goods
\begin{equation*}
  C_{i} = \bigg( \sum_{n=1}^{N} (C_{i}^{n})^{\frac{\sigma-1}{\sigma}}\bigg)^{\frac{\sigma}{\sigma-1}}
\end{equation*}
where $\sigma$ denotes the standard elasticity of substitution and $C_{i}^{n}$ denotes the supply of good $n$ in location $i$. Locations specialise in the production of goods such that each location only supplies one variety $n \in \mathcal{N}$. Let
\begin{equation*}
  Y_{i} = Y_{i}^{n}
\end{equation*}
denote the total production output of location $i$.\footnote{Note how this assumption does not normally coincide with perfect specialisation in the sense that every good is only produced in one location \citep[as in e.g.][]{Anderson_Gravitygravitassolution_2003}. This \emph{Armington} assumption is only satisfied if $N=I$ (and every good being produced).} In contrast to \citeauthor{fajgelbaum_optimal_2017}, my version of the model takes output levels as given and hence does not take a stand on how the output is produced. It is conceivable to think of a production function in which consumers produce goods themselves or an environment in which capital does the work for them. Importantly, however, by remaining agnostic about the production process, output levels are treated as fixed throughout the analysis. The only assumption about the supply side of the economy is perfect specialisation on one of the goods, which does not allow locations to choose between different production opportunities.

\subsection{Network Topography}
Locations $\mathcal{I}$ can be conceived as representing the nodes of an undirected network graph. Each location $i$ is directly connected to a set of neighbours $N(i) \in \mathcal{I} \setminus \{ i\}$. I consider locations to lie on a two-dimensional rectangular grid where each node is connected to its eight surrounding nodes to the North, North-East, East, and so on. Nodes at the border of the network graph might have fewer than eight neighbours. Let $\mathcal{E}$ denote the set of edges connecting neighbouring nodes and note that $(\mathcal{I}, \mathcal{E})$ fully describes the underlying network topography.\footnote{While I follow the notation of \cite{fajgelbaum_optimal_2017}, the set up so far is very similar to many of the recent contributions in the theory of trade in networks, like \cite{allen_welfare_2016} or \cite{Galichon_OptimalTransportMethods_2016}.}

All goods can be traded within the network. Let $Q_{i,k}^{n}$ denote the total flow of good $n$ travelling between nodes $i$ and $k \in N(i)$. The network affects trade flows in the sense that goods can only be shipped between neighbouring nodes. However, nothing prevents goods to travel far distances through the network by passing multiple locations after each other. To send a good on to a far away destination, the optimal trade route will potentially make multiple stop-overs in intermediate locations. The entire flow-topography of the trade network can hence be modelled simply by considering flows between neighbouring nodes.

Shipping goods from location $i$ to location $k \in N(i)$ incurs trade costs, which are modelled in the canonical iceberg form. In order for 1 unit of good $n$ to arrive at location $k$, origin location $i$ has to send $(1+\tau_{i,k}^{n})$ units on its way. $\tau_{i,k}^{n}$ can be understood as a tax on transport, causing a fraction of goods to not arrive in the destination location. This iceberg formulation has analytical advantages, as it makes modelling an entire transport sector superfluous. Importantly, barriers to trade are the dimension through which geography is introduced to the model. It is now costly for goods to travel elaborate and long routes through the network and notions of distance and connectedness start to play a role. The main assumption in \citeauthor{fajgelbaum_optimal_2017}'s derivation of optimal networks is then about the functional form of $\tau_{i,k}^{n}$. In particular, the authors model iceberg trade costs for shipping good $n$ between neighbouring locations $i$ and $k$ as
\begin{equation}
  \tau_{i,k}^{n} = \delta^{\tau}_{i,k} \frac{(Q_{i,k}^{n})^{\beta}}{I_{i,k}^{\gamma}}
  \label{eq:tau}
\end{equation}
where $I_{i,k}$ is defined as the level of \emph{infrastructure} on the edge between nodes $i$ and $k$. Since model parameters are restricted at $\delta^{\tau}_{i,k}, \beta, \gamma >0$, more infrastructure on a given link decreases the cost of trading between them. $I_{i,k}$ will later be calibrated more succinctly, but for now it suffices to picture anything that might make trade costs between two locations smaller, like broader and better roads, less detours or faster train tracks. In a first-best scenario, infrastructure between all nodes would be set infinitely high to wash away all trade costs and enable goods to travel seamlessly through the entire network. As $I_{i,k}$ is independent from $n$, higher infrastructure between two nodes will benefit the trade of all goods travelling on this edge. $\delta^{\tau}_{i,k}$ is a scaling parameter, which allows trade costs to be flexibly adjusted for any given origin-destination pair. Note that $\delta^{\tau}_{i,k}$ is also  independent of good $n$. In later calibrations, the main component of $\delta^{\tau}_{i,k}$ will be the geographical distance between locations $i$ and $k$. Distance is thought to increase trade costs independently from infrastructure provision on the travelled link.

The strongest assumption of the \citeauthor{fajgelbaum_optimal_2017} framework is the dependence of trade costs on $Q_{i,k}^{n}$, the total flow of goods on the link. Higher existing trade volumes on any given edge make sending an additional good more costly, a dynamic the authors refer to as \emph{congestion}. A few things are worth noting about this assumption. First, congestion plays the role of an externality in the trade network. Sending one additional unit of goods from $i$ to $k$ makes all other existing shipments more expensive. The social planner realises this and takes congestion into account when determining optimal trade flows.\footnote{Perhaps not surprisingly, the existence of a negative externality will lead to over-provision of trade in the decentralised market solution to the problem. In their technical appendix \citeauthor{fajgelbaum_optimal_2017} demonstrate how a moderately straightforward transport tax can internalise the externality and establish the applicability of the two welfare theorems. As this thesis stays in the social-planner conception of the problem, such decentralisation conditions are of minor concern going forward.} Second, congestion forces \emph{are} allowed to vary with different goods. While analytically appealing, this assumption might strike the reader as odd at first. Congestion for transporting some good $n$ is only caused by existing trade flows of the same good. No matter how many other goods are being pushed through an edge, each one only causes congestion nuisances for other shipments of the same variety. While \citeauthor{fajgelbaum_optimal_2017} demonstrate a way to circumvent this peculiarity of the model, as will be discussed soon the assumption has clear analytical advantages and is hence kept for the purpose of this thesis.

In equilibrium, each location cannot consume and export more than it produced and imported. More formally
\begin{equation}
  C_{i}^{n} + \sum_{k\in N(i)}^{}Q_{i,k}^{n}(1+\tau_{i,k}^{n}(Q_{i,k}^{n}, I_{i,k})) \leq Y_{i}^{n} + \sum_{j\in N(i)}^{}Q_{j,i}^{n}
  \label{eq:balanced_flows}
\end{equation}
must hold for every $n$ and $i$. Equation \eqref{eq:balanced_flows} is a variation of what \citeauthor{fajgelbaum_optimal_2017} call the \emph{Balanced Flows Constraint}.

\subsection{Optimal Infrastructure Provision}
So far, we have merely cast a plain model of trade in a geographical network, albeit with a slightly unconventional trade cost functional. Solving this model is more or less straightforward and would result in trade flows caused by heterogeneity in production endowments and population, but constrained by iceberg trade costs. The latter depend partially on the level of infrastructure present on any given node, which until now was taken as exogeneous. It is the main contribution of the \cite{fajgelbaum_optimal_2017} framework to endogeneise infrastructure provision $I_{i,k}$ in order to facilitate optimal trade flows. This allows for identifying areas in need of further infrastructure investment and computing their optimal level of expansion. Intuitively, a social planner would want to increase infrastructure on a link between locations which would heavily benefit from more mutual trade.

Analytically, this problem nests the static trade flow exercise mentioned above. The social planner chooses an infrastructure network, and given the network proceeds to compute the optimal trade flows subject to the \emph{Balanced Flows Constraint} \eqref{eq:balanced_flows}. In joint optimisation, it is the social planner's goal to construct a network in order to then induce optimal trade flows in the nested problem.

When designing the optimal network, the model a-priori leaves the possibility of differential infrastructure provision depending on the direction in which goods travel (that is $I_{i,k} \neq I_{k,i}$). However, in order to circumvent solutions in which some export-heavy nodes only have one-way streets leaving the location while consumption-heavy nodes cannot ever be left, I impose symmetry and restrict $I_{i,k} = I_{k,i} \textrm{ } \forall \textrm{ } i,k\in N(i)$.

Without a constraint on infrastructure provision, the problem is trivially non-sensible: the social planner would just provide each edge with an infinite amount of infrastructure in order to induce free shipment of goods and perfect consumption smoothing over locations. To make the problem more interesting, I follow \citeauthor{fajgelbaum_optimal_2017} in introducing a constraint on infrastructure. This is specified in fairly straightforward manner as the \emph{Network Building Constraint}
\begin{equation}
  \sum_{i}^{}\sum_{k\in N(i)}^{}\delta^{I}_{i,k}I_{i,k} \leq K
  \label{eq:network_building}
\end{equation}
where $\delta^{i}_{i,k}$ denotes the cost of building infrastructure on the edge between nodes $i$ and $k$. This cost is allowed to vary by link, such that one can model heterogeneity in infrastructure building cost caused by ruggedness, elevation, or the like. Total spending on infrastructure is constrained by $K$. One can think of $K$ as the total budget allotted to infrastructure expenditure, or the total amount of concrete available in the economy. It is taken as exogeneous and does not compete with other endowments in the network.\footnote{One could think of an extension in which $K$ has to be produced before being put on the road. The authors briefly discuss this possibility. To keep the model operational, I abstain from this extension and treat $K$ as exogeneous and fixed.}

In the application of this thesis, $K$ is interpreted as the total sum originally spent on building the \emph{existing} road network of a country. I observe the current road network of the economy, infer how much it must have cost to build it, and set $K$ equal to this amount. The social planner's task of choosing an optimal network while treating the total amount of concrete in the economy as fixed, hence amounts to a reallocation exercise. The social planner gathers all the concrete available and gets to redistribute it in a more sensible way. Improving infrastructure between two nodes in order to foster local trade comes at the cost of having to take away infrastructure someplace else. I argue that how much she has to rearrange existing interconnections serves as a sensible measure of spatial efficiency in the existing network. Before fully deriving and interpreting this measure, I first present the full planner's problem and ensuing equilibrium.

\subsection{Planner's Problem and Equilibrium}
In the nested problem, the social planner observes localities, endowments, population, and preferences and solves for trade flows between nodes that optimise overall welfare. She also solves for the optimal transport network which induces welfare-maximising trade flows in the nested problem while respecting the network building constraint \eqref{eq:network_building}. The full planner's problem can hence be stated as

\begin{subequations}
\begin{alignat*}{3}
&\!\max_{\substack{\big\{C_{i}^{n}, \{Q_{i,k}^{n}\}_{k\in N(i)}\big\}_{n}, \\ c_{i}, \big\{I_{i,k}\big\}_{k\in N(i)}}}        &\qquad &  \sum_{i}^{} L_{i}u(c_{i}) \qquad &\\
&\text{subject to} &\qquad & L_{i}c_{i} \leq \bigg( \sum_{n=1}^{N} (C_{i}^{n})^{\frac{\sigma-1}{\sigma}}\bigg)^{\frac{\sigma}{\sigma-1}} \qquad& \text{\textsc{\scriptsize \textbf{CES Consumption}}} \\
&                  &\qquad & C_{i}^{n} + \sum_{k\in N(i)}^{}Q_{i,k}^{n}(1+\tau_{i,k}^{n}(Q_{i,k}^{n}, I_{i,k})) \leq Y_{i}^{n} + \sum_{j\in N(i)}^{}Q_{j,i}^{n} \qquad& \text{\textsc{\scriptsize \textbf{Balanced Flows Constraint}}} \\
&                  &\qquad & \sum_{i}^{}\sum_{k\in N(i)}^{}\delta^{i}_{i,k}I_{i,k} \leq K \qquad& \text{\textsc{\scriptsize \textbf{Network Building Constraint}}} \\
&                  &\qquad & I_{i,k} = I_{k,i} \text{ for all } i\in \mathcal{I}, k\in N(i) \qquad& \text{\textsc{\scriptsize \textbf{Infrastructure Symmetry}}} \\
&                  &\qquad & C_{i}^{n}, c_{i}, Q_{i,k}^{n} \geq 0 \text{ for all } i \in \mathcal{I},n \in \mathcal{N},k\in N(i). \qquad& \text{\textsc{\scriptsize \textbf{Non-Negativity Conditions}}}
\end{alignat*}
\end{subequations}

My version of the planner's problem follows the baseline \cite{fajgelbaum_optimal_2017} model. However, four important differences need to be emphasised. First, in my model all goods are tradeable and no local amenities exist. Second, I do not allow workers to migrate between places and hence differences in marginal utility might still exist between nodes. Third, my model remains agnostic about the production function of each location and no analysis of the optimal use of input factors is undertaken. Fourth, I impose infrastructure symmetry. All these changes are undertaken with the later calibration and reshuffling exercise in mind.

Solving the planner's problem appears potentially daunting for two reasons. First, it is ex-ante not clear whether a unique optimum exists. Second, since one of the control variables is infrastructure $I_{i,k}$, even if a unique optimum were to exist, one might have to optimise over the very large space of possible networks, making the problem numerically intractable. Luckily, \cite{fajgelbaum_optimal_2017} provide conditions under which both these concerns can be allayed.

To show how a unique optimum can exist, first note that every constraint of the social planner's problem is linear but potentially for the balanced flows constraint. However, the introduction of congestion to iceberg trade cost causes even the balanced flows constraint to be convex if $\beta > \gamma$. Every part of the lengthy constraint is linear, but for the interaction term $Q_{i,k}^{n}\tau_{i,k}^{n}(Q_{i,k}^{n}, I_{i,k})$ representing total trade costs. Since $\tau_{i,k}^{n}$ was parameterised as in \eqref{eq:tau}, this expands to
\begin{equation}
  Q_{i,k}^{n}\tau_{i,k}^{n}(Q_{i,k}^{n}, I_{i,k}) = \delta^{\tau}_{i,k} \frac{(Q_{i,k}^{n})^{1+\beta}}{I_{i,k}^{\gamma}}
\end{equation}
which is convex if $\beta > \gamma$.\footnote{This is \citeauthor{fajgelbaum_optimal_2017} Proposition 1.} Under this condition, the social planner's problem is to optimise a concave objective over a convex set of constraints, guaranteeing the existence of a global optimum. $\beta > \gamma$ describes a notion of congestion dominance: increased infrastructure expenditure might alleviate the powers of congestion, but it can never overpower it. Intuitively, it precludes corner solutions in which all available concrete is spent on one link, all but washing away trade costs and leading to overwhelming transport flows on this one edge. If $\beta > \gamma$ geography always wins.

Second, instead of optimising over the large space of networks, \citeauthor{fajgelbaum_optimal_2017} harness the convexity result to instead solve the dual of the problem specified above. Instead of solving for every single infrastructure link, trade flows for all goods, and consumption patterns in each location, we can recast the problem as a set of first-order conditions from the subproblems, which only depend on Lagrange multipliers of each constraint. There are considerably fewer multipliers than primal control variables, namely one for every good in every node (interpretable as local prices). We are hence left to only find a price field from which under the convexity assumptions, all other properties follow. As \citeauthor{fajgelbaum_optimal_2017} note, solving the dual is common practice in optimal transport literature as it makes cumbersome large-scale optimisation problems much more tractable.\footnote{It is still a quite demanding  task to solve the ensuing dual problem, even numerically. Invoking duality reduces the scale of the problem, but we are still left with optimising over $I \times N$ variables.} I thus obtain the optimal network by constructing the Lagrangian corresponding to the planner's problem, deriving its first-order conditions, and recasting them as functions of the Lagrange parameters. These functions are largely equivalent to the ones derived in \citeauthor{fajgelbaum_optimal_2017}'s technical appendix, with the only differences attesting to the four changes discussed above. I can then reinsert these formulations into the original Lagrangian, which is now simply a function of the Lagrange parameters. Numerical optimisation now yields the solution to the dual problem. By inserting the parameters back into the derived first-order conditions, one can immediately derive the optimal infrastructure network $I_{i,k}$, optimal trade flows $Q_{i,k}^{n}$ over this network, and ensuing consumption patterns $C_{i}^{n}$ in each location.

Having laid out the network topography and properties of the planner's problem and having ensured that a unique solution exists and is realistically obtainable, I now proceed to calibrate the model with spatial data in order to derive a novel dataset of network efficiency for the entire African continent.
% End Chapter

\section{Bringing the model to the data -- Towards a spatial measure of African network inefficiency}
\label{chapter:calibration}
% Chapter
Analysing the spatial distribution of transport network inefficiency first requires an accurate representation of the \emph{current} topography of economic activity and infrastructure for every African country. I can then employ the \cite{fajgelbaum_optimal_2017} optimisation exercise outlined above to create a counterfactual transport network maximising overall welfare. Comparing the current network to its counterfactual, I can detect regions which would be granted additional infrastructure and places that would lose it. I can then derive a measure of network inefficiency over regions by comparing current welfare with welfare under the optimised scenario.

Below, I discuss these steps in more detail.

\subsection{Network Nodes}

To construct a spatial representation of all African countries, I divide the entire continent into grid cells of 0.5 degrees latitude by 0.5 degrees longitude (roughly 55 by 55 kilometres at the equator). For all of Africa, this amounts to 10,167 cells. Using GIS, I locate the geometric centroid of each cell and overlap it with current political borders to assign countries to each centroid.\footnote{Data on African borders come from \cite{Sandvik_WorldBordersDataset_2008}. Because of its relative age, this dataset does not include borders on South Sudan. I hence add the world's youngest country manually using data from \cite{OCHA_SouthSudanAdministrative_2017}. Taking the centroid position as the decisive statistic to assign countries might lead to situations where cells near oddly shaped borders are assigned country A, even if they mostly lie in country B. However, I don't believe this to be a major concern, especially since colonial legacies have led to African borders being drawn in a particularly straight-line fashion \citep[see][]{Alesina_ArtificialStates_2011}. I also occasionally use data on the coordinates of capital cities which I take from \href{http://techslides.com/list-of-countries-and-capitals}{\texttt{ techslides.com/list-of-countries-and-capitals}}.} I then use spatial data on economic and geographic characteristics from a variety of sources and aggregate them onto this grid cell level.

Raster data on 2015 population totals come from the \textit{Gridded Population of the World} dataset \citep[GPW,][]{socioeconomic_data_and_applications_center_gridded_2016}. This NASA-funded project gathers data from hundreds of local census bureaus and statistical agencies in order to construct a consistent high-resolution spatial dataset of the world's population. When a datasource only reports population totals for large, higher-level administrative districts, the dataset smoothes population uniformly over the entire area. Importantly, it does \emph{not} employ any auxiliary data sources (like, crucially, satellite data) to weight-adjust population totals in those situations \citep{Doxsey-Whitfield_TakingAdvantageImproved_2015}.\footnote{The dataset's methodology also streamlines older population estimates over time by extrapolating each observation by its national growth rate to the year 2015 \citep{Doxsey-Whitfield_TakingAdvantageImproved_2015}.} Africa is the continent with the coarsest resolution of administrative input data. However, the average coverage of $57\textrm{KM}^{2}$ neatly matches the grid cell resolution of my study. I hence overlay this GPW data with my 10,000+ grid cells to obtain the total number of people living in each cell. On average, a cell is home to 110,000 people, with the median much to the left of that (25,000). The most populous cell contains Cairo and inhabits almost 18 million people. 212 cells are uninhabited.

Geographic characteristics include altitude, soil quality, average temperature, precipitation, yearly growing days, malaria prevalence, and terrain ruggedness. These data come from \cite{henderson_global_2018} and are also aggregated to the slightly coarser resolution of my study.\footnote{Note that \cite{henderson_global_2018} use a slightly different measure of terrain ruggedness compared to the earlier work by \cite{nunn_ruggedness:_2012}. All my later findings are, however, robust to using either measure.} \citeauthor{henderson_global_2018} show that these indicators account for a substantial part of the global variation of economic activity and hence serve as an important set of controls for later empirical examinations.

Lastly, to proxy for heterogeneities in economic activity over space, I rely on the established practise of using satellite imagery of light intensity at night. Satellites from the \textit{Defense Meteorological Satellite Program} take high-resolution pictures of the entire globe between 8:30pm and 10:00pm every night. Following the seminal work by \cite{henderson_measuring_2012}, more lit places have since been associated with more economic activity or growth in numerous settings, including ethnic homelands \citep{michalopoulos_national_2014}, cities \citep{storeygard_farther_2016}, or the Korean peninsula \citep{Lee_InternationalIsolationRegional_2016}.\footnote{See \cite{donaldson_view_2016} for an excellent review.} I use pre-processed data on night luminosity from \cite{henderson_global_2018} which was captured by satellites in 2010. This dataset bears an improvement over other existing night lights data, as it is able to better discern differences at the very right tail of the light distribution and hence prevents many problems related to top-coding of the highest-lit places. Taking means, this data is also aggregated to the 0.5 x 0.5 degree resolution. Well-known measurement problems related to blooming and overglow (where highly lit places may make their neighbours appear brighter than they are) should not be a major concern on this rather coarse resolution \citep{michalopoulos_spatial_2018}. Night lights are approximately log-normal distributed and hence have to be converted into logarithmic form when serving as the dependent variable in standard regression settings. However, in my study, lights serve as a calibration parameter for differences in economic output and thus do not have to be transformed.

Luminosity at night is highly correlated with population density. In fact, it is not immediately clear whether higher-lit places are richer, more productive, or just more populous. Setting up my trade model requires both economic output and population to be independent measures. When disentangling the two factors, the GPW dataset's strict reliance on census data comes in handy. No satellite measure is used to distribute population totals over large enumeration areas, curtailing the threat of counting anything twice. Furthermore, while night lights and population are both certainly reported with some measurement error, these should not be systematically related. Census bureaus generate error through non-scientific questionnaires and negligent reporting, whereas satellites generate error through technological malfunction. These sources of measurement imprecision should not be related and hence not bias the results \citep[see also][]{Pinkovskiy_LightsCameraIncome_2016}.\footnote{There is, in fact, fairly high variation in night-lights per capita \citep[a measure which should be employed with scientific caution and is only reported here for illustrative purposes, see][]{michalopoulos_spatial_2018}. The 90th percentile produces 360 times as many lights per capita as the 10th percentile of grid cells.}

\subsection{Network Edges}
So far, the dataset merely consists of a list of grid cells with their respective characteristics. To gain a conception of their relative position in a trade network, a measure of connectedness between locations is needed. In particular, one needs to know whether two locations are connected at all, how strong the link between the two already is, how costly it is to improve the link, and how costly it is to trade between the two. Or, in the notation of \cite{fajgelbaum_optimal_2017}, an infrastructure matrix $I_{i,k}$, an infrastructure investment cost parameter $\delta_{i,k}^{I}$, and a trade cost parameter $\delta_{i,k}^{\tau}$.

\subsubsection{Road Data}
\label{sec:road_data}
Obtaining objective data on transportation networks is difficult, especially in developing countries where many transport routes are not available in digitised form. In their own empirical exercise \citeauthor{fajgelbaum_optimal_2017} observe European countries and make use of a large coherent dataset on position and objective characteristics of important European roads. A comparable dataset for Africa does not yet publicly exist, even though a recent project by \cite{jedwab_average_2017} has undertaken the effort to manually compile and digitise Michelin Maps in order to create a comparable dataset (their data is not yet available for replication).

Instead, I take a different approach and make use of the open source internet routing service \textsc{Open Street Maps} (OSM), which is comparable to Google Maps but allows for unlimited use of its API. For every centroid location, I scan OSM for the optimal route to each of their respective eight surrounding neighbours. This is greatly facilitated by the R package \texttt{osrmRoute}.\footnote{Scans of OSM were conducted in November 2017. The service does not allow a retrospective scan over past road databases, so a time difference between lights (2010), population (2015), and roads (2017) can not be overcome. If anything, one could argue, this renders my network inefficiency measure a lower bound to its true value since government officials will have had time to adjust their network to any spatial economic imbalances. If the 2017 network only inefficiently supports 2010-15 trade, chances are it did even worse in 2015.} Since I am interested solely in within-country transport networks, I perform the exercise for each country separately and do not elicit connections between locations of different countries. Hence, centroids located near a coast or country border often have less than eight immediate neighbours. For all of the resulting almost 90,000 routes, I gather distance travelled, average speed, and step-by-step coordinates of the travel path.

% Figure
\begin{figure}[t]
\centering
\caption{Road Networks for different countries as scraped off OSM}

\begin{subfigure}[c]{0.43\textwidth}
\includegraphics[width=\textwidth, trim={1cm 1cm 0cm 2cm},clip]{/Users/Tilmanski/Documents/UNI/MPhil/Second Year/Thesis_Git/Build/output/Road_Networks/network_Nigeria.png}
\caption{Nigeria}
\label{fig:nigeria_roads}
\end{subfigure}
\begin{subfigure}[c]{0.43\textwidth}
\includegraphics[width=\textwidth, trim={1cm 1cm 0cm 2cm},clip]{/Users/Tilmanski/Documents/UNI/MPhil/Second Year/Thesis_Git/Build/output/Road_Networks/network_Mali.png}
\caption{Mali}
\label{fig:Mali_roads}
\end{subfigure}

\begin{subfigure}[c]{0.43\textwidth}
\includegraphics[width=\textwidth, trim={1cm 1cm 0cm 2cm},clip]{/Users/Tilmanski/Documents/UNI/MPhil/Second Year/Thesis_Git/Build/output/Road_Networks/network_Ethiopia.png}
\caption{Ethiopia}
\label{fig:Ethiopia_roads}
\end{subfigure}
\begin{subfigure}[c]{0.43\textwidth}
\includegraphics[width=\textwidth, trim={1cm 1cm 0cm 2cm},clip]{/Users/Tilmanski/Documents/UNI/MPhil/Second Year/Thesis_Git/Build/output/Road_Networks/network_Rwanda.png}
\caption{Rwanda}
\label{fig:Rwanda_roads}
\end{subfigure}


\label{fig:roads}
\end{figure}
% End Figure

The OSM routing algorithm is specified for cars and takes into account differential speeds attainable on different types of roads. However, if either start or destination location do not directly fall onto a street, the optimal route jumps to the nearest road and goes from there. To take this into account, I add a walking distance to the travel path. Agents are assumed to walk in straight lines to the nearest street at a fixed speed of 4 km/h. They then take the car and drive the route with average speed as specified by OSM, before they potentially have to walk the last stretch again to their exact centroid destination. For some particularly remote areas, even the nearest street is very far away, such that the car routing provided by OSM is not sensible. To counter these cases, I also calculate for all 90,000 connections the outside option of walking the entire link in a straight line at 4 km/h. I then identify cases in which walking directly is actually faster than using OSM's proposed route (plus the travel to and from roads). In these cases, I replace OSM's route with the walking distance and (constant 4 km/h) speed. Figure \eqref{fig:roads} presents the resulting road networks for four African countries. Figure \eqref{fig:nigeria_roads} displays every optimal route for Nigeria, which appears overall fairly well connected. Commuters mostly seem to be able to drive relatively direct routes between locations, even though cases with substantial detours are also evident at second glance. Connections in which walking were the preferred alternative are displayed in red and fairly rare in Nigeria. Figure \eqref{fig:Mali_roads} presents the case of Mali, which paints a different picture: for many connections through the Sahara desert in the North-East of the country, walking straight lines in the sand is actually the fastest way to get from A to B. Ethiopia in Figure \eqref{fig:Ethiopia_roads} displays only a few trails connecting the country's East to the West. Small Rwanda in Figure \eqref{fig:Rwanda_roads} zooms in on the actual roads taken and displays the intricacies of the optimal routing provided by OSM.

Relying on the open source community of OSM does come with some drawbacks. Most importantly, data on the position and quality of roads is user-generated and hence subject to reporting bias. Intuitively, richer areas may appear to be equipped with more roads if local residents have the time and necessary access to a computer to enter their neighbourhoods into the database. As soon as inference is conducted on the relationship between streets and any covariate of development, the resulting estimates will be biased. While this is certainly troubling, I believe this bias to be much more important on finer resolutions than the operating one in this study. Start and destination of the elicited routes are on average more than 55 kilometres apart and travel will hence take place mostly on larger roads and national highways. It is unlikely that these major streets are systematically underreported in OSM, the primary open source routing platform on the Internet, especially compared to the alternative of digitised Michelin maps. Reporting bias will definitely be an issue when trying to find the optimal route \emph{within} a particular small neighbourhood in Accra, but the OSM database should do a fairly good job in finding the optimal route between Accra and Kumasi. It is nevertheless important to keep this potential flaw of the data in mind when conducting inference later on.\footnote{There is a second, less troubling problem with using OSM data. As this study merely pertains to within-country transport networks, I only look at connections between neighbouring locations of the same country. However, in some cases the optimal route between these locations still might go through a neighbouring country. For instance, Senegal is effectively split into two parts by the intersecting country The Gambia. Still, when connecting Senegalese cell centroids just to the north and to the south of The Gambia (which are still less than 60km apart), the route will go over foreign soil. This presents a problem only in later policy recommendations, as political leaders of one country cannot necessarily legislate road improvements abroad. The rest of the analysis is not affected by this issue.}

In some rare cases (less than 0.1 per cent of all connections), the OSM algorithm cannot find any route between two neighbouring centroid locations. This mostly is due to an obvious geographic impossibility to connect two nodes. In Guinea-Bissau, for instance one location lies on the Bolama Islands just off the shore of mainland Guinea-Bissau. Its neighbouring locations are all on the mainland and hence unreachable by car. In other cases, both locations to be connected are in deep jungle or swampy regions. In all these cases, I treat the link as if the two locations were not neighbours in the first place. That implies I even forgo the backup possibility of walking the entire distance, assuming that agents cannot walk between islands or through the densest jungle.

After having collected data on distance and average speed on the optimal route between all neighbouring centroid locations, the next step is to discretise these data in order to have a tractable network representation capable of performing the trade simulations necessary in the remainder of this study.

\subsubsection{Infrastructure matrix $I_{i,k}$}
To gain a conception of how much two given nodes are connected in the transport network, one needs to derive a numeric measure of how much \emph{current} infrastructure lies on the link between locations. In their own empirical analysis, \cite{fajgelbaum_optimal_2017} have data on the average number on lanes of the streets used on a given route, and whether these streets are national or secondary roads. The OSM algorithm does not supply this detailed level of information for Africa. However, I argue that the authors are only proxying for a much more immediate statistic: the average speed with which one can travel on a given road. If two locations are linked by a faster connection, I assume this to be the result of higher infrastructure $I_{i,k}$ on this edge. Obviously there are many factors influencing driving speed other than number of lanes or road classifications. Congestion, altitude differences, or potholes come to mind. However, what the authors are trying to capture is an infrastructure investment vehicle to reduce trade costs. Certainly building more lanes on a given road will reduce trade costs. But it will do so by increasing the speed with which cars can travel on that road. I argue that observing the average speed with which transport can occur between two nodes is a much more immediate measure of how well these nodes are connected. I hence propose
\begin{equation}
  I_{i,k} = \textrm{Average Speed}_{i,k}
\end{equation}

This measure is naturally bound from below at 4 km/h, as walking the air-line distance is always available as a backup. Empirically, average speeds range between 6 km/h (Mauritania, where most of the distances through the desert have to be covered by walking) and 33 km/h (Swaziland). It is now the objective of the policymaker or social planner to reduce trade costs between suitable trading partners by increasing the average speed $I_{i,k}$ with which transport can occur between them.

\subsubsection{Trade cost parameter $\delta_{i,k}^{\tau}$}
Recall from equation \eqref{eq:tau} that iceberg trade costs between nodes $i$ and $k$ are modelled as $\tau_{i,k}^{n} = \delta^{\tau}_{i, k} \frac{(Q_{i,k}^{n})^{\beta}}{I_{i,k}^{\gamma}}$. Following \citeauthor{fajgelbaum_optimal_2017} and ensuring convexity and strong duality, I parameterise $\beta = 1.245$ and $\gamma = 0.5\beta = 0.6225$.

$\delta^{\tau}_{i, k}$ is a scaling parameter. It impacts trade costs regardless of current infrastructure levels and congestion forces and hence allows for variation caused by exogeneous forces like the geography of a given edge. \citeauthor{fajgelbaum_optimal_2017} calibrate is as a linear function of the distance between $i$ and $k$ and in order to match some particular patterns from Spanish trade data. Their calibration is not applicable to the given context. Firstly, it deals with European developed countries which recent evidence suggests face much less stifling trade costs than the African continent \citep[see e.g.][]{Anderson_Tradecosts_2004}. Secondly, their calibration revolves around different measures for economic output $Y_{i}$ and infrastructure investment $I_{i,k}$ and will hence produce arbitrary estimates for $\delta^{\tau}_{i, k}$.

Instead, I make use of a recent contribution by \cite{atkin_whos_2015}. They use barcode-level data on sale prices of identical goods to back out trade costs between regions within both Nigeria and Ethiopia. They show that trade costs are significantly increasing in distance between origin and destination, stifling much of mutually beneficial trade. In comparison to a similar exercise in the US, the authors again provide evidence that trade costs in Africa are exceptionally dependent on distance travelled. Directly taking the average of their two point estimates for Ethiopia and Nigeria, I calculate
\begin{equation}
  \delta^{\tau}_{i,k} =  0.466\times\textrm{ln}(\textrm{Distance}_{i,k})
  \label{eq:delta_tau}
\end{equation}

as the trade cost elasticity to distance travelled.\footnote{\citeauthor{atkin_whos_2015} (Table 2, page 44) estimate this parameter as $0.0374$ for Ethiopia and $0.0558$ for Nigeria. My parameter is the simple average of these two point estimates.} Note that the functional form of \eqref{eq:tau} still implies that the very first good to be shipped over a given link is free of any trade costs, with convex congestion forces coming into play  afterwards. The social planner can now invest into infrastructure along any link to attenuate these forces.

\subsubsection{Infrastructure building cost matrix $\delta_{i,k}^{I}$}
The \emph{Network Building Constraint} \eqref{eq:network_building} binds the social planner's action space when commissioning faster roads. Total cost of infrastructure $\sum_{i}^{}\sum_{k \in N(i)}^{} \delta_{i,k}^{I}I_{i,k}$ is fixed at $K$, which equals the cost of originally building the current network. In this set-up, $\delta_{i,k}^{I}$ denotes the relative constant cost of increasing the average speed on a given link by one. As $K$ is unrelated to the rest of the model it can be normalised to $K=1$ without loss of generality, so that only relative infrastructure building costs matter. This procedure allows for staying agnostic about the actual (dollar-)cost of building roads and only needs to make a stance on which areas are cheaper or more expensive to build on \emph{relative to others}.

$\delta_{i,k}^{I}$ depends on a variety of inputs, like the distance of a road or the underlying terrain. I follow \citeauthor{fajgelbaum_optimal_2017} who in turn make use of a recent study by \cite{collier_cost_2015} which estimates infrastructure building costs in the developing world. Readily applying their specification, I calculate
\begin{equation}
  \textrm{ln}(\delta^{I}_{i,k,c}) = \delta_{c}^{I} - 0.11 \times \one_{\textrm{Distance}_{i,k,c} > 50km} + 0.12 \times \textrm{ln}(\textrm{Ruggedness}_{i,k,c}) + \textrm{ln}(\textrm{Distance}_{i,k,c})
  \label{eq:delta_I}
\end{equation}

as the constant cost of increasing infrastructure $I_{i,k}$ on the link between $i$ and $k$ in country $c$. $\textrm{Distance}_{i,k,c}$ denotes the road distance travelled between nodes and enters positively, implying that longer roads are costlier to develop as every single road kilometre will have to be improved. Moreover, the building cost per kilometre falls discretely when the distance surpasses 50 kilometres, as embodied by the indicator $\one_{\textrm{Distance}_{i,k,c} > 50km}$. Note that every route in my sample is longer than 50 kilometres and the corresponding dummy term is hence always equal to 1. $\textrm{Ruggedness}_{i,k,c}$ denotes the average ruggedness between grid cells $i$ and $k$ and enters positively, highlighting the additional expenses accompanied with building on uneven terrain. $\delta_{c}^{I}$ is a country-specific scaling parameter. Its main purpose it to ensure that equation \eqref{eq:network_building} is satisfied with $K=1$. I hence appraise the infrastructure network $I_{i,k}$ of all countries and then flexibly alter $\delta_{c}^{I}$ for each nation individually in order to comply with equation \eqref{eq:network_building}.

It is worth noting that $\textrm{Distance}_{i,k,c}$ denotes the distance travelled along the fastest route as described in section \ref{sec:road_data}, \emph{not} the air-line distance between origin and destination centroid. This is significant in the sense that distance is treated as a primitive of the model and not endogeneously subject to the optimisation exercise. If under the current network two nodes are only connected via an extensive detour, the distance variable as reported by Open Street Maps will be very large. The social planner is now merely equipped with the capacity to improve the infrastructure on the given route, but cannot devise an entirely new road shortcutting the detour. This restriction might very well be reasonable in certain situations. Some nodes might be only connected by a rather circuitous route because of a major geographical obstacle, mountains or swamps, between them. My conception of infrastructure investment ensures that no improbable road through a mountain is devised, but rather the path around the mountain is improved. On the other hand, my model might forgo obvious welfare-improving infrastructure projects based on entirely new roads. In the abstract formalism of the model, this distinction does not make that much of a difference. It can afford to stay agnostic about whether trade costs are decreased by a new shortcut or by road improvements on the old connection. When using the predictions of the model to  inform actual transport policy, however, one has to bear in mind that entirely new roads might at times succeed in connecting regions at even lower costs than the improvements prescribed by my simulation.

\subsection{Heterogeneous goods}
To build incentives for trade, I introduce $N=2$ different goods: an agricultural and an urban good. Since economic output is proxied by night luminosity, I cannot observe the distribution of different goods in a given grid cell, but only their total production. This leads to the total specialisation assumption outlined above in which urban grid cells are solely producing the urban good, while all other grid cells are producing nothing but the agricultural good.

To classify grid cells as urban or rural I use an iterative procedure, which seeks to match each country's 2016 urbanisation rate as reported by \cite{the_world_bank_world_2017}. I start by assuming every location is a city and then gradually proceed to re-classify the least densely populated locations, until the ratio of people living in urban areas to total population equals that of the World Development Indicators.\footnote{For three countries, the WDI do not report urbanisation rates. In these cases, I match the overall urbanisation rate for the entire African continent of 42 per cent as reported by \cite{lall_africas_2017}.} With this procedure, seven per cent of grid cells are classified as urban. These cells inhabit 40 per cent of the continent's population, matching recent figures from \cite{lall_africas_2017} fairly well.

As discussed above, the model stays agnostic about how grid cells produce their luminosity output. This is in contrast to \citeauthor{fajgelbaum_optimal_2017} who allow for endogenous labor allocation across locations and goods and hence need to take a stance on functional forms of total factor productivity and output. The only assumption my model makes is that cells can ever only produce one of the two varieties and their productivity in the respective other variety is zero. I also avoid having to divide total lights by total population in order to obtain TFP, a procedure which has occasionally been used \citep[see e.g.][]{DeLuca_Ethnicfavoritismaxiom_2018}, even though the literature has recently moved towards consensus of warning against it \citep{michalopoulos_spatial_2018}.

\begin{figure}[t]
\centering
\caption{Discretised Networks for different countries}

\begin{subfigure}[c]{0.45\textwidth}
\includegraphics[width=\textwidth, trim={2cm 1cm 1.5cm 0cm},clip]{/Users/Tilmanski/Documents/UNI/MPhil/Second Year/Thesis_Git/Build/output/Matlab_graphs/Nicer_graphs/Nigeria_stat.png}
\caption{Nigeria}
\label{fig:nigeria_mat}
\end{subfigure}
\begin{subfigure}[c]{0.45\textwidth}
\includegraphics[width=\textwidth, trim={2cm 1cm 1.5cm 0cm},clip]{/Users/Tilmanski/Documents/UNI/MPhil/Second Year/Thesis_Git/Build/output/Matlab_graphs/Nicer_graphs/Mali_stat.png}
\caption{Mali}
\label{fig:Mali_mat}
\end{subfigure}

\begin{subfigure}[c]{0.45\textwidth}
\includegraphics[width=\textwidth, trim={2cm 1cm 1.5cm 0cm},clip]{/Users/Tilmanski/Documents/UNI/MPhil/Second Year/Thesis_Git/Build/output/Matlab_graphs/Nicer_graphs/Ethiopia_stat.png}
\caption{Ethiopia}
\label{fig:Ethiopia_mat}
\end{subfigure}
\begin{subfigure}[c]{0.45\textwidth}
\includegraphics[width=\textwidth, trim={2cm 1cm 1.5cm 0cm},clip]{/Users/Tilmanski/Documents/UNI/MPhil/Second Year/Thesis_Git/Build/output/Matlab_graphs/Nicer_graphs/Rwanda_stat.png}
\caption{Rwanda}
\label{fig:Rwanda_mat}
\end{subfigure}
\label{fig:matlab_networks}
\end{figure}

\subsection{Simulation}

After these steps, a discretised network representation now exists for every African country. Nodes in the network are the spaced centroid locations of each grid cell. They combine the characteristics of the entire grid cell (population, output, etc.) in one point. Edges in the network are road connections between centroids. Each edge carries a number of characteristics (average speed, trade costs, and infrastructure building costs). Figure \eqref{fig:matlab_networks} presents this discretised network specification for the four countries from above. Nodes are printed larger proportional to their population. Edges are drawn thicker proportional to the initial infrastructure investment (i.e. average attainable speed).

For each country, I proceed to conduct two simulation exercises. For both, I calibrate the curvature parameter of the utility function at $\alpha = 0.4$ and the elasticity of substitution parameter at $\sigma=4$.\footnote{The parametrisation of $\alpha$ is identical in \cite{fajgelbaum_optimal_2017}. They, however, calibrate elasticity of substitution at $\sigma=5$, so slightly higher than my parameter. \cite{Head_GravityEquationsWorkhorse_2014} review 32 related studies and find a mean parameter of $\sigma=4.51$. This average is, however, driven by some positive outliers as the median parameter is much to the left of that. To account for this slant, I choose $\sigma=4$.} In the first simulation exercise, infrastructure $I_{i,k}$ is treated as fixed. This is to obtain a baseline estimate of the spatial variation of welfare in each country. Locations are still allowed to trade with each other, but only over the exogeneous current road network. Formally, this corresponds to solving a slightly truncated version of the social planner exercise from above, where $I_{i,k}$ is simply dropped from the planner's set of control variables and exogeneously set to the empirical infrastructure matrix $I_{i,k}^{\textrm{empirical}}$.\footnote{Recall that the optimal network allocation exercise nests the problem of solving for optimal trade flows. Hence, every other aspect of the model remains unaffected by fixing $I_{i,k}$. Trivially, this makes the \emph{Network Building Constraint} \eqref{eq:network_building} non-binding. The same outcome could be achieved at much higher computational costs by simply introducing an additional constraint $I_{i,k} = I_{i,k}^{\textrm{empirical}}$ to the social planner's problem.} By construction, the resulting solution will have two properties. Firstly, total output over the entire country will remain untouched. Inputs are not defined and hence do not shift to more productive regions. Indeed, any welfare gains will be attained solely by shipping the right mix of goods to the right regions. Secondly, labor immobility will leave welfare differences between regions as agents cannot simply move to more privileged cells. The social planner would like to overcome these differences, but is confronted with trade costs which might leave certain remote areas much worse off than well-connected ones.

Following this static exercise, I proceed to the main task of endogeneising the infrastructure matrix $I_{i,k}$. With the \emph{Network Building Constraint} binding total infrastructure investment at the level of the current road network, the social planner is now free to reshuffle roads within the country in order to improve connections as she chooses. If she wants to improve the connection between two given locations, she will have to take away infrastructure from somewhere else in the country. Intuitively, this reallocation exercise does not seek to identify where to place the optimal next investment, but rather represents an utterly fictitious scenario in which every road can be lifted from the ground, reshuffled, and eventually located someplace else.\footnote{Note that equation \eqref{eq:network_building} only fixes $\sum_{i}^{}\sum_{k \in N(i)}^{} \delta_{i,k}^{I}I_{i,k} = K$. Hence, not the overall sum of infrastructure is fixed, but more precisely the overall cost of infrastructure. This still allows the social planner to take away one unit of infrastructure on a very expensive (high $\delta_{i,k}^{I}$) link and exchange it for much more than one unit on a cheaper (low $\delta_{i,k}^{I}$) link.} The procedure does not measure how many roads a country has, but rather how well they are placed. It does not look at whether the entire country is full of speedy roads, but rather whether those roads connect the right locations.

I conduct the reallocation scenario for every African country. Five small countries (Cape Verde, Comoros, The Gambia, Mauritius, and Reunion) are too small to form a sensible network as they only show up as a single location in the dataset and are henceforth no longer considered. Computation times are greatly diminished when exploiting the strong convexity of the optimisation setting and solving the dual problem as sketched in the technical appendix of \cite{fajgelbaum_optimal_2017}. Optimisations are performed via Matlab's \texttt{fmincon} command. When conducting the simulations, I bind the social planner's set of permissible roads from below, at 4 km/h (such that $I_{i,k} \geq 4$ $ \forall \textrm{ } i,k$). This is motivated by the assumption at the beginning that walking straight lines at this speed is conceived as an outside option and always available to any commuter. The social planner should not be able to force commuters to travel slower than walking, in order to build a faster road elsewhere.\footnote{Contrarily, I do not explicitly restrict possible investments from above (at least not in addition to the sum-restriction imposed by Equation \eqref{eq:network_building}), as this could violate the strong convexity of the problem. Not bounding the problem in principle allows the social planner to combine every available infrastructure from all over the country into one supersonic speed highway on one particular node. However, the model is calibrated in a way which makes this very unattractive to the planner anyway. After simulating reallocation in every African country, less than 0.8\% of all built roads were suggested to be over 260 km/h. Still, one outlier of 2007 km/h (in Egypt) and one of 1755 km/h (in South Africa) remain.}

\begin{figure}[!h]
\centering
\caption{Reallocation Scenario for different countries}

\begin{subfigure}[c]{0.45\textwidth}
\includegraphics[width=\textwidth, trim={2cm 1cm 1.5cm 0cm},clip]{/Users/Tilmanski/Documents/UNI/MPhil/Second Year/Thesis_Git/Build/output/Matlab_graphs/Nicer_graphs/Central-African-Republic_stat.png}
\caption{Central African Republic, pre reallocation}
\label{fig:cae_pre}
\end{subfigure}
\begin{subfigure}[c]{0.45\textwidth}
\includegraphics[width=\textwidth, trim={2cm 1cm 1.5cm 0cm},clip]{/Users/Tilmanski/Documents/UNI/MPhil/Second Year/Thesis_Git/Build/output/Matlab_graphs/Nicer_graphs/Central-African-Republic_opt.png}
\caption{Central African Republic, post reallocation}
\label{fig:cae_post}
\end{subfigure}

\begin{subfigure}[c]{0.45\textwidth}
\includegraphics[width=\textwidth, trim={2cm 1cm 1.5cm 0cm},clip]{/Users/Tilmanski/Documents/UNI/MPhil/Second Year/Thesis_Git/Build/output/Matlab_graphs/Nicer_graphs/United-Republic-of-Tanzania_stat.png}
\caption{Tanzania, pre reallocation}
\label{fig:tanzania_pre}
\end{subfigure}
\begin{subfigure}[c]{0.45\textwidth}
\includegraphics[width=\textwidth, trim={2cm 1cm 1.5cm 0cm},clip]{/Users/Tilmanski/Documents/UNI/MPhil/Second Year/Thesis_Git/Build/output/Matlab_graphs/Nicer_graphs/United-Republic-of-Tanzania_opt.png}
\caption{Tanzania, post reallocation}
\label{fig:tanzania_post}
\end{subfigure}

\begin{subfigure}[c]{0.45\textwidth}
\includegraphics[width=\textwidth, trim={2cm 0cm 1.5cm 0cm},clip]{/Users/Tilmanski/Documents/UNI/MPhil/Second Year/Thesis_Git/Build/output/Matlab_graphs/Nicer_graphs/Rwanda_stat.png}
\caption{Rwanda, pre reallocation}
\label{fig:rwanda_pre}
\end{subfigure}
\begin{subfigure}[c]{0.45\textwidth}
\includegraphics[width=\textwidth, trim={2cm 0cm 1.5cm 0cm},clip]{/Users/Tilmanski/Documents/UNI/MPhil/Second Year/Thesis_Git/Build/output/Matlab_graphs/Nicer_graphs/Rwanda_opt.png}
\caption{Rwanda, post reallocation}
\label{fig:rwanda_post}
\end{subfigure}

\label{fig:Reallocations}
\end{figure}

Figure \eqref{fig:Reallocations} visualises this reallocation exercise for several countries. Subfigure \eqref{fig:cae_pre} displays the discretised network representation of the Central African Republic, comparable to Figures \eqref{fig:nigeria_mat} -- \eqref{fig:Rwanda_mat}. The edges to this network are printed almost evenly thick, implying that infrastructure is fairly evenly distributed across the country. Subfigure \eqref{fig:cae_post} then displays the country after the network reshuffling exercise. Three patterns stand out. First, the social planner sees a clear need to connect the populous areas in the South West of the country with each other. Some southern nodes are granted extensive, almost highway-like connections to their immediate neighbours. For that, the social planner is willing to salvage some of the apparently unnecessary infrastructure in the middle or north of the country. Second, there still seems to be a benefit from having a few trails connecting the South West with the North East. Some clear North-South and East-West trails spanning multiple regions emerge. Thirdly, nodes are printed in a colour scale corresponding to individual welfare gains and losses for each location. As can be seen from first-glance, most southern regions stand to gain between five and ten per cent of total welfare from this scenario. Hardly any nodes seem to lose welfare, even though on second glance a few instances become apparent.

Tanzania in Figures \eqref{fig:tanzania_pre} -- \eqref{fig:tanzania_post} displays a more decentralised optimal network solution. The reallocation scenario results in the main urban areas being better connected to their immediate surroundings, but no clear overarching network seems to emerge. There also does not appear to be any necessity to connect hinterland regions with the primal city Dar es Salaam in the West. Indeed, the largest city slightly loses welfare with the reallocation at the expense of multiple smaller population centres in the North.\footnote{On a sidenote, Tanzania also illustrates an interesting case where a tiny fraction of the country is fully detached from the rest of the network. Just north of Dar es Sallam, the island of Zanzibar constitutes a one-node subnetwork of its own. Not surprisingly, it remains completely unaffected by the reshuffling of roads on the mainland. Instances like these are relatively common in the dataset.} Small Rwanda in Figures \eqref{fig:rwanda_pre} -- \eqref{fig:rwanda_post} helps illustrating some of the forces at hand in a less crowded graph. Starting from a fairly evenly distributed transport network, the reallocation dynamics lead to much more variation in infrastructure provision. Some links are deemed superfluous and hence reduced to the smallest admissible level, while others are scaled multiple times their starting infrastructure level. Furthermore, high welfare gains are reported by direct neighbours of big production centres and urban grid cells. These are unassuming grid cells with average population or output levels, merely equipped with the geographical blessing of being close to a bigger neighbour. This leads to the conclusion that while better infrastructure combats the welfare costs of geographical distance, proximity to hubs still matters. Even an optimally designed transport network is ultimately not able to overcome the \emph{curse of distance}.\footnote{A term coined by \cite{Boulhol_Havedevelopedcountries_2010}.}

After successfully reshuffling a country's transport network, overall welfare will necessarily (weakly) increase. It is the social planner's objective to maximise overall welfare, and since the original network composition is always still available, the entire country cannot on aggregate be worse off than before. Note again that overall production (light output) will be unaffected by the entire exercise. Welfare gains are solely caused by enabling mutual benefits from trade through connecting the right locations. Nevertheless, they are substantial. The Central African Republic of Figure \eqref{fig:Reallocations}, for instance, stands to gain 1.84\% of overall welfare just by reshuffling its roads. Tanzania (1.7\%) and Rwanda (1.27\%) are slightly closer to their hypothetical optimum.

\begin{figure}
\centering
\caption{African countries by network inefficiency}
\includegraphics[width=0.6\textwidth,trim={1cm 4cm 1cm 3cm},clip]{/Users/Tilmanski/Documents/UNI/MPhil/Second Year/Thesis_Git/Analysis/output/zeta_heatmaps/African_countries_zeta.png}

\label{fig:countries_by_welfare_gain}
\mysubcaption{Map of the African continent. Countries coloured according to their welfare gain under the optimal reallocation counterfactual. Scale from almost 7\% (dark red) to $<1\%$ (white) welfare gains. Gains are computed by comparing the population-weighted sum of utility levels over all grid cells in a country before and after the reallocation exercise. As labor is immobile, the measure perfectly corresponds to per capita utility gains as well.}
\end{figure}

Figure \eqref{fig:countries_by_welfare_gain} displays all African countries and their hypothetical welfare gain. The three countries considered above perform rather well in comparison. Some (mostly more developed) countries like South Africa (0.5\%) or Tunisia (0.2\%) perform even better. Many countries are leaving much more on the table, like Somalia (4.8\%) or Chad (4.3\%). No African country, however, has a more ill advised road network than South Sudan. Its citizens stand to gain almost 6.7\% of overall welfare if just their roads were better placed. This might not come as a surprise, as the world's newest country has largely inherited a road network that was not conceived to sustain an independent nation, but rather connect it to its former capital up north.

Forgone welfare gains can be conceived as an intuitive measure for overall network inefficiency. The closer hypothetical gains to zero (the lighter the country's colour), the more efficient the current allocation of roads. Vice-versa, if a country stands to gain a lot from reshuffling, then the current network is deemed more inefficient. On a simple cross-section, countries with less efficient networks are significantly correlated with more corruption ($p < 0.01$), less property rights ($p < 0.01$) and less 2010 log GDP ($p=0.07$). Note that these are merely descriptive correlations which are far from implying any form of causation.\footnote{Data from \cite{the_world_bank_world_2017}. For corruption and property rights, data is only available for 35 countries and correlations are hence performed on this truncated sample. Interestingly, network efficiency is \emph{not} statistically associated with earlier independence years ($p=0.8$) or more artifical border designs ($p=0.3$) as reported by \cite{Alesina_ArtificialStates_2011}.}

While each country only stands to gain overall welfare from this reallocation procedure, individual locations might very well lose in the process. Intuitively, some regions might be equipped with far too many good roads such that the social planner takes these roads away to use someplace else. Comparing each grid cell's welfare before and after the major reshuffling can help identifying regions which are currently over or under-provided for. More formally, I define

\begin{equation}
  \Lambda_{i} = \frac{\textrm{Welfare under the optimal Infrastructure}_{i}}{\textrm{Welfare under the current Infrastructure}_{i}}
\end{equation}

% Figure
\begin{figure}[t]
\centering
\caption{Spatial Distribution of $\Lambda_{i}$ for sample countries}

\begin{subfigure}[c]{0.32\textwidth}
\includegraphics[width=\textwidth]{/Users/Tilmanski/Documents/UNI/MPhil/Second Year/Thesis_Git/Analysis/output/zeta_heatmaps/Central-African-Republic_zeta.png}
\caption{Central African Republic}
\label{fig:Central African Republic_zeta}
\end{subfigure}
\begin{subfigure}[c]{0.32\textwidth}
\includegraphics[width=\textwidth]{/Users/Tilmanski/Documents/UNI/MPhil/Second Year/Thesis_Git/Analysis/output/zeta_heatmaps/United-Republic-of-Tanzania_zeta.png}
\caption{Tanzania}
\label{fig:Tanzania_zeta}
\end{subfigure}
\begin{subfigure}[c]{0.32\textwidth}
\includegraphics[width=\textwidth]{/Users/Tilmanski/Documents/UNI/MPhil/Second Year/Thesis_Git/Analysis/output/zeta_heatmaps/Rwanda_zeta.png}
\caption{Rwanda}
\label{fig:Rwanda_zeta}
\end{subfigure}

\begin{subfigure}[c]{0.32\textwidth}
\includegraphics[width=\textwidth]{/Users/Tilmanski/Documents/UNI/MPhil/Second Year/Thesis_Git/Analysis/output/zeta_heatmaps/Madagascar_zeta.png}
\caption{Madagascar}
\label{fig:Madagascar_zeta}
\end{subfigure}
\begin{subfigure}[c]{0.32\textwidth}
\includegraphics[width=\textwidth]{/Users/Tilmanski/Documents/UNI/MPhil/Second Year/Thesis_Git/Analysis/output/zeta_heatmaps/Kenya_zeta.png}
\caption{Kenya}
\label{fig:Kenya_zeta}
\end{subfigure}
\begin{subfigure}[c]{0.32\textwidth}
\includegraphics[width=\textwidth]{/Users/Tilmanski/Documents/UNI/MPhil/Second Year/Thesis_Git/Analysis/output/zeta_heatmaps/Chad_zeta.png}
\caption{Chad}
\label{fig:Chad_zeta}
\end{subfigure}



\label{fig:zeta_countries}
\mysubcaption{Six African countries by their Local Infrastructure Discrimination Index $\Lambda_{i}$ on the grid cell level. Maps show each country as the 0.5 x 0.5 degree grid used for the network optimisation. For each map, darker shaded cells correspond to higher $\Lambda_{i}$ levels and hence more infrastructure discrimination compared to optimal network. To better visualise within-country variation, colour scale slightly changes from country to country.}
\end{figure}
% End Figure

as the \emph{Local Infrastructure Discrimination Index} for grid cell $i$. Areas with high $\Lambda_{i}$ scores ($\Lambda_{i} > 1$) would be gaining under the optimal reallocation scenario and are hence under provided for in the network's current state. A score of $\Lambda_{i} < 1$ on the other hand, implies that a region is \emph{too} well off given its position in the network and hence should be stripped of some of its infrastructure to increase overall welfare. Figure \eqref{fig:zeta_countries} displays the spatial distribution of $\Lambda_{i}$ for six countries. The darker a grid cell's shade, the more it is disadvantaged by the inefficiencies of the current network. Figures \eqref{fig:Central African Republic_zeta} -- \eqref{fig:Rwanda_zeta} display what could already be inferred from the colouring of the nodes in Figure \eqref{fig:Reallocations}: The Central African Republic has too little roads in the South West and too many in the East, Tanzania shows no clear spatial pattern, and Rwanda only engages in minor reshuffling. In Figures \eqref{fig:Madagascar_zeta} -- \eqref{fig:Chad_zeta} it, furthermore, becomes apparent that Madagascar's infrastructure network discriminates against the island's heartland (note how the coastal areas tend to be much lighter than the hinterland), Kenya would profit from connecting Nairobi to its surroundings, and Chad's South is discriminated against compared to the North.



% Figure
\begin{figure}
\centering
\caption{Spatial Distribution of $\Lambda_{i}$ for entire sample}
\includegraphics[width=0.65\textwidth,trim={1cm 4cm 1cm 5cm},clip]{/Users/Tilmanski/Documents/UNI/MPhil/Second Year/Thesis_Git/Analysis/output/zeta_heatmaps/African_gridcells_zeta.png}

\label{fig:all_gridcells_by_zeta}
\mysubcaption{Africa as represented by 10,167 grid cells of 0.5 x 0.5 degrees. Cells coloured according to their $\Lambda_{i}$ index of local infrastructure discrimination. Darker cells would benefit from reallocating national infrastructure networks. Note that the hypothetical reallocation scenario is conducted on the country-level. Cells from different countries are hence not immediately comparable. Maps's colouring follows an equal-interval rule such that every colour in the spectrum has an equal amount of members. This is to visualise the measure's variation but leads to unequal bracket-sizes for each colour.}
\end{figure}
% End Figure

Figure \eqref{fig:all_gridcells_by_zeta}, lastly, displays the spatial variation of $\Lambda_{i}$ over all 10,000+ grid cells of the entire African continent. When interpreting this map, note that grid cells are undergoing the reshuffling scenario solely within their respective country. National borders hence play a role and can at times even clearly be inferred from the printed map.\footnote{There are two reasons why I conduct the simulation procedure within countries and not over the entire African continent. One is computational; the requirements for numerically solving the model increase quadratically in the number of locations $\mathcal{I}$. The largest country in Africa (Algeria) is made up of almost 900 locations and already strains computing power quite heavily. Simulating all of Africa's 10,000+ locations at once is then almost unattainable with available technology. The second reason is interpretational; while lifting a country's roads from the ground and flexibly reshuffling them across the nation is already a fictitious scenario, it still operates within a government transport authority's locus of control. Regions disadvantaged by their own government can reasonably be considered discriminated against. This is less the case if one were to optimise over the entire continent. Without a central planning body for all of Africa, it is hard to interpret why a road in e.g. Tunisia should rather be moved into Namibia.}  Keeping this in mind, the map reveals substantial spatial variation in the infrastructure gap across the African continent. The luckiest region (in Namibia) stands to loose almost 30\% of total welfare if the fictitious social planner intervened and reshuffled roads away from it. On the other extreme of the spectrum, the residents of one grid cell in Gabon are missing out on a welfare hike of more than 50\%. Moreover, abandoned regions are clearly displaying spatial correlation with large neighbouring swaths of land collectively missing out on infrastructure improvements in certain countries. This begs the conclusion that countries do not just overlook single grid cells but rather live with vast stretches of disadvantaged regions. The index is evidently representing more than just haphazard noise.

In the following sections, I proceed to analyse patterns behind this heterogeneity of network infrastructure discrimination over space.
% End Chapter

\section{Empirical strategy and results}
% Chapter
Why are some African roads in the wrong place? To investigate which areas have too much or too little infrastructure, I employ the \emph{Local Infrastructure Discrimination Index} $\Lambda_{i}$ as dependent variable in a standard OLS regression setting. In the base specification, I estimate
\begin{equation}
  \Lambda_{i,c} = \beta v_{i,c} + \textbf{X}_{i,c}\gamma + \delta_{c} + \epsilon_{i,c}
  \label{eq:grid_ols}
\end{equation}
for a variety of different independent variables $v_{i,c}$ of grid cell $i$ in country $c$. $\delta_{c}$ denotes country-fixed effects, $\textbf{X}_{i,c}'$ is a vector of controls, and $\beta$ is the coefficient of interest. The dependent variable $\Lambda_{i}$ roughly follows a normal distribution and hence no transformation is undertaken. However, as seen in Figure \eqref{fig:all_gridcells_by_zeta}, local infrastructure discrimination clearly displays autocorrelation over space causing the error term $\epsilon_{i,c}$ to not be distributed independently. To account for this problem, I follow \cite{Bester_Inferencedependentdata_2011} and construct a higher-level spatial grid of 3 degrees latitude by 3 degrees longitude and cluster standard errors within each of these higher-level grid cells. Errors are  allowed to covary within each cluster, but not between them. This technique draws its power from constructing clusters in the most arbitrary manner possible without relying on potentially endogenous partitions like national borders or administrative units \citep[see e.g.][]{michaels_resetting_2017}.\footnote{There are 332 such clusters. Since 3 degrees is evenly divisible by the observation-level grid cell size of 0.5 degrees, each cluster in principle fits 36 observations. The median cluster does indeed comprise 36 cells, but some border-regions fit fewer observations. On average, there are 31 observations in a cluster.}

Including country-fixed effects is of particular importance, as $\Lambda_{i,c}$ is constructed by optimising trade flows within each country separately. A grid cell in Egypt is hence not directly comparable to one in Sierra Leone. Country-fixed effects account for this underlying heterogeneity and make observations comparable internationally. In addition, the vector of controls $\textbf{X}$ captures observable characteristics of each grid cell which might plausibly account for some of the variation in $\Lambda_{i,c}$. \cite{henderson_global_2018} show that a surprisingly parsimonious set of geographical and agricultural covariates explains a substantial part of the global variation in economic activity. Making use of their data, I include in $\textbf{X}$ each grid cell's average altitude, average temperature and precipitation, land suitability for agriculture, length of the annual growing period, and an index for the stability of malaria transmission. I also include mutually exclusive (and collectively exhaustive) dummy variables classifying each grid cell into one of twelve predominant vegetation regions \citep[or \emph{Biomes}, see][]{henderson_global_2018}. To flexibly account for any broad geographic trend over the entire continent, I additionally add fourth-order polynomials of both latitude and longitude for each grid cell. To take into account that some regions have a natural advantage in conducting trade, I also include indicators for whether a grid cell's centroid is within 25 kilometres of a natural harbor, big lake, or navigable river respectively. Lastly, I create a dummy for whether a cell is at the border of a country's trade network and has less than eight immediate neighbours.

I call the set of controls outlined so far \emph{``Geographic Controls''}. They are in principle unaffected by human decisions about the design of trade networks and therefore plausibly exogeneous. Another set of covariates, however, poses more difficulties. These are the variables that were already used to calibrate the optimal reallocation simulation from above, namely a cell's population, light output, ruggedness, and classification into urban and rural.\footnote{Recall that population and output were crucial components of the planner's problem, ruggedness went int the cost of building new infrastructure $\delta^{I}_{i,k}$, and the urban/rural classification determined which good a cell produced.} I call these \emph{``Simulation Controls''}. It is important to be aware that $\Lambda_{i}$ is, among others, already a product of the intricate interplay between these factors. There is hence a danger for plain OLS to detect a spurious, mechanical relationship between them, potentially biasing results. On the other hand, not controlling for the spatial distribution of people and economic activity creates the risk of confounding estimates by means of omitted variable bias. For instance, if colonial railroads were built to connect cities and $\Lambda_{i}$ tends to be higher for urban areas, naive OLS will falsely identify a relationship between railroads and network inefficiency that is not actually there. To confront this dilemma, I always report estimates with and without the set of simulation controls. As I will demonstrate, results turn out to be largely similar between the two, hinting at the highly non linear genesis of $\Lambda_{i}$.\footnote{Since they are partially determined by variables from the set of geographic controls, population and night lights can also be seen as \emph{``bad controls''} in the sense of \cite{Angrist_MostlyHarmlessEconometrics_2008}. They warn against controlling for any covariate that could also be interpreted as an outcome of some of the other regressors. Again, I counter this potential fallacy by always reporting estimates with and without the set of simulation controls.}

My measure of network inefficiency only pertains to systems of goods trade. As discussed above, however, there are clearly other rational motivations for building roads, mainly facilitating the commute of people to large administrative hubs. To ensure that my results are not driven by systematic ignorance of these not-for-trade roads, I re-estimate every model while excluding grid cells containing a country's capital. As reported below, this leaves results virtually unchanged.

In this thesis, I investigate three potential sources of network inefficiency in Africa: colonial-era infrastructure investments, ethnic power relations, and foreign aid. Each individual setting faces different challenges to identification but they are all based on the framework outlined above. The following sections present empirical strategies and results for all three strands of inquiry.

\subsection{Colonial Infrastructure Investments}
At the Berlin Conference of 1884--1885, European powers unilaterally decided to split up the African continent into spheres of influence amongst themselves, setting off what is known as the \emph{Scramble for Africa}. The colonising powers rushed to secure areas of control, formed local administrative hierarchies, and set up extractive economies depriving the continent of its natural resources. The following decades went on to transform the continent in a multitude of dimensions. A large strand of literature has investigated the long-run effects of colonial policies on contemporary comparative development in Africa, including institutional reforms \citep{Acemoglu_ColonialOriginsComparative_2001,acemoglu_reversal_2002}, border design \citep{michalopoulos_long-run_2016}, and missionary education \citep{Wantchekon_EducationHumanCapital_2015}.

The colonial powers also transformed the landscape of many African regions by devising large scale infrastructure projects. Starting in the late 19th century, numerous railway lines were built to facilitate the transport of goods and troops through the vast newly appropriated territories. Between 1890 and 1960, British, French, Belgian, German, Italian, and Portuguese administrations all undertook efforts to permeate their colonies with more or less sophisticated railway networks \citep{jedwab_permanent_2016}. There were two main motivations for this: supporting the extractive economies and ensuring military domination \citep{jedwab_history_2017}. On the one hand, rail tracks were used to transport crops and minerals from the fields and mines of the African heartland to the nearest harbour city from which they were shipped back to Europe. On the other hand, geopolitical considerations led the colonial governments to rapidly expand their network to showcase their military might and secure their spheres of influence.

In Nigeria, for example, the British colonial administration was under pressure to counter increasing advances by the French into the northern parts of the country. In response,  three major railway lines were built to connect the North with the coast \citep{Falola_historyNigeria_2008}. However, these were used not only for military purposes, but also to transport ground nuts, palm oil, and cocoa to the harbours in Lagos, Port Harcourt, and Calabar \citep{ekundare_economic_1973}. Transporting goods between the various regions of the country had been relying on head porterage for centuries, rendering interregional trade in commodities all but impossible.\footnote{Until the construction of colonial railways, the only systematic interregional trade in many African countries had been the slave trade. Over five centuries leading up to the Berlin Conference about 18 million Africans were enslaved and traded out of the country, the majority by ship to the New World \citep{nunn_long-term_2008}.} Trips on the newly constructed lines were slow and arduous compared to modern standards, but nevertheless marked the first time goods and people could relatively easily travel between the different parts of the country. As a result, Britain secured their strategic grip on Nigeria's north, extracted minerals and crops on a large scale, and saw export volumes skyrocket \citep{chaves_reinventing_2013,woltjer_economic_2018}. Though their economic gains hardly ever benefited the local population, the construction of railroads could be seen as a transformative event for the Nigerian economy.\footnote{Other case studies for Ghana \citep{jedwab_permanent_2016}, Kenya \citep{jedwab_history_2017}, India \citep{Donaldson_RailroadsRajEstimating_2018}, or the US \citep{donaldson_railroads_2016,swisher_reassessing_2017} describe similarly transformative impacts of railroad construction on the local economy at the time.}

Apart from boosting the development of trade and migration at the time, colonial railroads have also been found to have a persistent impact on the spatial organisation of economic activity today. \cite{jedwab_permanent_2016} show how urbanisation started to center around railway tracks in the decades following their construction. Even as most railway lines have fallen into disrepair and road traffic has replaced trains as the most important means of transport, economic activity today still clusters in places close to the former rail lines. \citeauthor{jedwab_permanent_2016} see this as evidence that other spatial equilibria could have emerged, but railway locations facilitated equilibrium selection.

Is the current equilibrium efficient? To investigate whether railways from the colonial period still have an impact on trade network inefficiency today, I overlay the 10,000+ grid cells of my data set with every railway line built by the colonial powers in Sub-Saharan Africa. Figure \eqref{fig:Railroad_Map} prints in red all lines built between 1890 and the various independence dates. Data on railroad positioning comes from \cite{jedwab_permanent_2016}, with the exception of South Africa, for which I manually digitise a map from \cite{Herranz-Loncan_publicbenefitRailways_2017}. No comparable data is available for Madagascar, Egypt, and the Maghreb countries, which also saw some colonial railway construction. As discussed below, findings are robust to excluding grid cells from these countries.

For every grid cell, I compute the total number of colonial railway kilometres crossing the cell. This serves as tangible measure for the stock of physical transport capital invested into each region. The majority of cells (91\%) are not crossed by a colonial railway and hence have zero railroad kilometres. Those that are crossed by a line usually have between 20 and 60 railroad kilometres, while some important rail crossings or transport hubs have up to 100 kilometres of railroads (in a 55 by 55 kilometre cell nevertheless). While this measure captures the intensive margin of colonial infrastructure investment, I also construct a measure of extensive margin railroad exposure by computing the distance from a cell's centroid to its closest track. Furthermore, every railroad line comes with a classification of being constructed primarily for military purposes or mining purposes (or neither, or both), allowing for more nuanced analysis. Lastly, to account for potential endogeneity concerns, I also compute the same statistics for a set of railway lines the colonisers planned, but never built. As \citeauthor{jedwab_permanent_2016} explain, these projects were not realised only for a series of arguably random historical events like unforeseeable cuts to financing, the outbreak of wars, or sudden retirements of adminstration officials. If any effects are to be attested to the construction of railroads during the colonial era, no impact should be found for these \emph{placebo} railroads. These tracks are printed in blue in Figure \eqref{fig:Railroad_Map}.\footnote{Data for placebo lines also come from \cite{jedwab_permanent_2016} and \cite{Herranz-Loncan_publicbenefitRailways_2017}.}

% Figure Maps of Rails
\begin{figure}
\centering
\caption{Colonial Railway Network}

\begin{subfigure}[c]{0.48\textwidth}
\includegraphics[width=\textwidth,trim={10cm 11cm 6cm 10cm},clip]{/Users/Tilmanski/Documents/UNI/MPhil/Second Year/Thesis_Git/Analysis/output/other_maps/all_rails.png}
\caption{Colonial Rails (red) and Placebo Rails (blue)}
\label{fig:Railroad_Map}
\end{subfigure}
\begin{subfigure}[c]{0.48\textwidth}
\includegraphics[width=\textwidth,trim={10cm 11cm 6cm 10cm},clip]{/Users/Tilmanski/Documents/UNI/MPhil/Second Year/Thesis_Git/Analysis/output/other_maps/emst.png}
\caption{EMST Lines and Buffer}
\label{fig:EMST_Map}
\end{subfigure}

\label{fig:Rail Maps}
\mysubcaption{Maps displaying the network of railway lines and placebo railroads. Data from \cite{jedwab_permanent_2016} and \cite{Herranz-Loncan_publicbenefitRailways_2017}. Subfigure (a) prints in red  railroads built by the colonial powers between 1890 and 1960. Lines that were initially planned but never actually built are printed in blue. Subfigure (b) prints the euclidean minimum spanning tree (EMST), connecting the initial urban network of 1900 (red dots) with the least amount of total rail kilometres \citep[see ][]{jedwab_permanent_2016}. This is the rail system that would have minimised total infrastructure costs if the colonial powers had cooperated and jointly devised an ideal network. Subfigure (b) also prints a buffer of 40km around the EMST.}
\end{figure}
% End Figure

% Figure Table of RailKM_zeta
\begin{table}[t] \centering
  \caption{Colonial Railroads and Local Infrastructure Discrimination Index}
  \label{tab:RailKM_zeta}
  \resizebox{\textwidth}{!}{

  \begin{tabular}{@{\extracolsep{5pt}}lcccccccc}
  \\[-1.8ex]\hline
  \hline \\[-1.8ex]
   & \multicolumn{8}{c}{\textit{Dependent variable:}} \\
  \cline{2-9}
  \\[-1.8ex] & \multicolumn{8}{c}{Local Infrastructure Discrimination Index $\Lambda_{i}$} \\
  \\[-1.8ex] & (1) & (2) & (3) & (4) & (5) & (6) & (7) & (8)\\
  \hline \\[-1.8ex]
   KM of Colonial Railroads & $-$0.0002$^{***}$ & $-$0.0001$^{***}$ & $-$0.0002$^{***}$ & $-$0.0002$^{***}$ &  &  &  &  \\
  & (0.0001) & (0.0001) & (0.0001) & (0.0001) &  &  &  &  \\
    & & & & & & & & \\
   KM of Colonial Placebo Railroads &  &  &  &  & 0.00004 & $-$0.0002 & $-$0.0002 & $-$0.0003 \\
  &  &  &  &  & (0.0003) & (0.0003) & (0.0003) & (0.0003) \\
    & & & & & & & & \\
  \hline \\[-1.8ex]
  Country FE &  & Yes & Yes & Yes &  & Yes & Yes & Yes \\
  Geographic controls &  &  & Yes & Yes &  &  & Yes & Yes \\
  Simulation controls &  &  &  & Yes &  &  &  & Yes \\
  Observations & 10,158 & 10,158 & 10,158 & 10,158 & 10,158 & 10,158 & 10,158 & 10,158 \\
  R$^{2}$ & 0.001 & 0.099 & 0.124 & 0.126 & 0.00000 & 0.098 & 0.122 & 0.124 \\
  \hline
  \hline \\[-1.8ex]
  \textit{Note:}  & \multicolumn{8}{r}{$^{*}$p$<$0.1; $^{**}$p$<$0.05; $^{***}$p$<$0.01} \\
  \end{tabular}

}

\mysubcaption{Results of estimation of equation \eqref{eq:grid_ols} on the sample of 0.5x0.5 degree grid cells for the entire African continent (excluding five small countries, see text). Dependent variable is the Local Infrastructure Discrimination Index for each grid cell. Columns (1)-(4) estimate the effect of colonial infrastructure investments as measured by the total number of colonial railroad kilometres crossing a cell. Starting with a simple univariate cross-section in (1), column (2) adds 49 country-fixed effects. Column (3) adds geographic controls, consisting of altitude, temperature, average land suitability, malaria prevalence, yearly growing days, average precipitation, indicators for the 12 predominant agricultural biomes, indicators for whether a cell is within 25 KM of a natural harbor, navigable river, or lake, the fourth-order polynomial of latitude and longitude, and an indicator of whether the grid cell lies on the border of a country's network. Simulation controls are added in column (4) and are comprised of population, night lights, ruggedness, and a dummy for whether a cell is classified as urban. These are indicators that went into the original infrastructure re-allocation simulation and are hence not orthogonal to $\Lambda$. Columns (5)-(8) repeat these calculations with railroads that were planned, but never built (``placebo railroads''). Results are robust to using only the subsample of 33 countries with colonial infrastructure investment as reported by \cite{jedwab_permanent_2016}, plus South Africa (not reported). Results are also robust to excluding all grid cells containing a country's capital (not reported). Heteroskedasticity-robust standard errors are clustered on the 3x3 degree level and are shown in parantheses.}
\end{table}
% End Figure

Table \eqref{tab:RailKM_zeta} displays results from OLS estimation of Equation \eqref{eq:grid_ols} with $\Lambda_{i,c}$ on the left hand side and rail kilometres as explanatory variable $v_{i,c}$. Column (1) displays the plain cross-sectional relationship without controls and reveals a statistically significant negative association between the two variables. Grid cells with high colonial railroad investment also have significantly lower infrastructure discrimination today. Recall that low values of $\Lambda_{i,c}$ correspond to regions losing welfare if the social planner were to optimally reallocate infrastructure. On a merely descriptive level, the negative estimate of column (1) hence implies that regions with colonial railroads crossing through them would on average see infrastructure (and welfare) being redistributed to those areas without colonial investments. This plain correlational relationship is far from implying any form of causation. A score of other variables could instead be responsible for this observational association, most prominently population. As \citeauthor{jedwab_permanent_2016} show, areas with colonial railroads are more populous today. If the social planner were to redistribute infrastructure from urban to rural areas, for example, any estimated relationship between railways and $\Lambda_{i,c}$ would be spurious. To investigate these reservations, columns (2) -- (4) gradually extend the set of observable controls. The point estimate and statistical significance of the persistence of railroads, however, is robust to including country fixed effects, geographical controls, and simulation controls as described above. In the richest specification of column (4), every ten kilometres of colonial railway construction lead grid cells to lose $0.002$ percentage points of welfare at the hand of areas without any investment.

The colonial authorities did not place railroads randomly. Decisions on where to invest in expensive transport infrastructure is rather subject to cost-benefit analyses, regional economic potential, and projections about which areas are expected to thrive. This poses a threat to identification as the construction of railroads does not constitute a perfect natural experiment. Rather, columns (1) -- (4) of Table \eqref{tab:RailKM_zeta} estimate the joint effect of a region being selected as a site for railroad investment \emph{together} with the actual investment. If the selection process relies on unobservables characteristics which impact $\Lambda_{i,c}$ today, estimates will be biased.\footnote{One could make the case that this bias might mask a much stronger effect in the same direction as in Table \eqref{tab:RailKM_zeta}. If railroads were built in areas which were expected to do well in the future, these regions might be receiving more infrastructure from the social planner today. This would attenuate the negative effect in columns (1) -- (4). However, other potential directions for the bias are conceivable as well.} To isolate the effect of actual infrastructure investments, columns (5) -- (8) repeat the exercise for the set of placebo railroads. If these are assumed to have undergone the same planning process as the actual railways, these regressions reveal the distorting power of site selection. None of the estimates are significantly different from zero, suggesting that the link in columns (1) -- (4) is a causal one. The results are virtually identical when using only the subsample of 34 countries for which data on colonial railroad placement is available (not reported).

The effects described in Table \eqref{tab:RailKM_zeta} are small, yet remarkable. Across the African continent, areas that received large infrastructure investments a century ago are still \emph{too} well off given their position in the national trade network. In contrast, areas that were not crossed by tracks are inefficiently short on infrastructure today. To see that this is a non-trivial finding, note that, firstly, most of the colonial railway lines have been in disrepair for decades and thus do not immediately dictate trade flows today. Secondly, recall that the optimal network reallocation and construction of $\Lambda_{i,c}$ was based on roads and cars, not rails and trains. The implication is hence \emph{not} that colonial railway systems themselves are inadequate to efficiently sustain inter-regional trade today. Rather the transport revolution a century ago coordinated the entire economy into a certain spatial equilibrium, which persists even though it has become inefficient. African nations would benefit from moving to a better equilibrium, but are locked in the current state. The placement of colonial railroads set in motion a process of spatial sorting with people, infrastructure, and output clustering in locations which are suboptimal today. The social planner tries to overcome these misallocations and move infrastructure away from regions once considered important by the colonisers. To summarise, \citeauthor{jedwab_permanent_2016} show that colonial investments helped the economy to coordinate on one of many spatial equilibria, while my findings suggest that this is not the optimal one.\footnote{Note that the spatial equilibrium induced by colonial railroads could have still been optimal \emph{at the time}. My argument solely concerns the persistent effects of investments a century ago on network efficiency \emph{today}.}

% Figure Table RailKM_mining_military
\begin{table}[t] \centering
  \caption{Heterogeneous Effects of Colonial Railroads}
  \label{tab:RailKM_mining_military}
  \resizebox{\textwidth}{!}{


  \begin{tabular}{@{\extracolsep{5pt}}lcccccccc}
  \\[-1.8ex]\hline
  \hline \\[-1.8ex]
   & \multicolumn{8}{c}{\textit{Dependent variable: Local Infrastructure Discrimination Index $\Lambda_{i}$}} \\
  \cline{2-9}
  \\[-1.8ex] & (1) & (2) & (3) & (4) & (5) & (6) & (7) & (8)\\
  \hline \\[-1.8ex]
   KM of Rails & $-$0.0002$^{***}$ & $-$0.0002$^{***}$ &  &  & $-$0.0002$^{**}$ & $-$0.0002$^{**}$ &  &  \\
   \hspace*{3mm} for Military Purposes & (0.0001) & (0.0001) &  &  & (0.0001) & (0.0001) &  &  \\
    & & & & & & & & \\
   KM of Rails  &  &  & $-$0.0001 & $-$0.0001 & $-$0.0001 & $-$0.0001 &  &  \\
  \hspace*{3mm} for Mining Purposes &  &  & (0.0001) & (0.0001) & (0.0001) & (0.0001) &  &  \\
    & & & & & & & & \\
    KM of Rails &  &  &  &  &  &  & $-$0.0003$^{***}$ &  \\
  \hspace*{3mm} within EMST Buffer &  &  &  &  &  &  & (0.0001) &  \\
     & & & & & & & & \\
    KM of Rails  &  &  &  &  &  &  &  & $-$0.0001$^{**}$ \\
  \hspace*{3mm} outside EMST Buffer &  &  &  &  &  &  &  & (0.0001) \\
     & & & & & & & & \\
  \hline \\[-1.8ex]
  Country FE & Yes & Yes & Yes & Yes & Yes & Yes & Yes & Yes \\
  Geographic controls & Yes & Yes & Yes & Yes & Yes & Yes & Yes & Yes \\
  Simulation controls &  & Yes &  & Yes &  & Yes & Yes & Yes \\
  Observations & 10,158 & 10,158 & 10,158 & 10,158 & 10,158 & 10,158 & 10,039 & 10,039 \\
  R$^{2}$ & 0.123 & 0.125 & 0.122 & 0.124 & 0.123 & 0.125 & 0.126 & 0.125 \\
  \hline
  \hline \\[-1.8ex]
  \textit{Note:}  & \multicolumn{8}{r}{$^{*}$p$<$0.1; $^{**}$p$<$0.05; $^{***}$p$<$0.01} \\
  \end{tabular}

}

\mysubcaption{Replication of estimations of Table \eqref{tab:RailKM_zeta} in estimating effects of colonial railroads on the Local Infrastructure Discrimination Index $\Lambda$. Colonial rails are classified as built for military or mining purposes (or neither or both) by \cite{jedwab_permanent_2016}. Columns (7)-(8) distinguish between railway kilometres within and outside the 40km-wide buffer around the euclidean minimum spanning tree (EMST) of the initial urban network. Geographic controls consist of altitude, temperature, average land suitability, malaria prevalence, yearly growing days, average precipitation, indicators for the 12 predominant agricultural biomes, indicators for whether a cell is within 25 KM of a natural harbor, navigable river, or lake, the fourth-order polynomial of latitude and longitude, and an indicator of whether the grid cell lies on the border of a country's network. Simulation controls are comprised of population, night lights, ruggedness, and a dummy for whether a cell is classified as urban. Results are robust to using only the subsample of 34 countries with colonial infrastructure investment, with only the estimate in column (8) estimated less precisely (but still $p<0.1$). Results are also robust to excluding all grid cells containing a country's capital (not reported). Heteroskedasticity-robust standard errors are clustered on the 3x3 degree level and are shown in parantheses.}
\end{table}
% End Figure

Which railways were responsible for coordinating African economies into a suboptimal spatial equilibrium? To understand the forces behind this dynamic, I split the sample of all colonial railways by their initial construction purpose. As discussed above, most tracks were built with military or mining considerations in mind. Similarly to the exercise above, I separately calculate the total number of mining and military railroad kilometres respectively. Table \eqref{tab:RailKM_mining_military} repeats the estimations from above for both subsets of railways. As can be seen from columns (1) -- (6), the effect is exclusively driven by railways built for military purposes. The social planner seeks to take welfare away from regions which were crossed by lines built for military domination. Railroads constructed to support the mining trade are not associated with areas too well off today. This distinction holds both when including the variables separately (columns 1-4) and jointly (columns 5-6).

This finding offers an intuitive understanding of how the current spatial distribution is inefficient. All colonial infrastructure investments spurred urbanisation close-by, regardless of their construction purpose. Since mining lines arguably cross areas that are still important trade routes today, the social planner sees no need to reorganise economic activity away from them. Areas surrounding rails built with military consideration in mind, however, have since lost their strategic importance. As the nature of military domination, state authority, and conflict have changed since the 19th century, there is no immediate value of clustering economic activity and infrastructure close to former military lines anymore. It is those sunk investments that skew the spatial equilibrium towards an inefficient state.

To further investigate the notion of differentiating railroad lines by their strategic importance, I follow \cite{jedwab_permanent_2016} and \cite{faber_trade_2014} and construct the euclidean minimum spanning tree (EMST) for the African urban network in 1900. This is the network which minimises the total euclidean distance of edges while ensuring that all nodes are connected to each other. Following \citeauthor{jedwab_permanent_2016}, I take as nodes all African cities with a population of over 10,000 people in 1900.\footnote{This amounts to 119 nodes. 105 of these are reported by \cite{jedwab_permanent_2016}, 14 are in South Africa and from a map in \cite{Herranz-Loncan_publicbenefitRailways_2017}.} Figure \eqref{fig:EMST_Map} prints the cities and the tree connecting them. I then construct a 40 kilometre buffer around the EMST lines and split all colonial railways by whether or not they are located within this buffer.\footnote{The choice of 40 kilometres is arbitrary and directly follows \cite{jedwab_permanent_2016}.} Intuitively, rails within the buffer have more strategic military importance than those outside it. Railroads built to support regional trade do not necessarily have to be on the straight line between major cities -- lines with the aim of quickly transporting troops do.

Columns (7) -- (8) of Table \eqref{tab:RailKM_mining_military} estimate the heterogeneous effects of both kinds of railway kilometres. Following \citeauthor{jedwab_permanent_2016}, I drop grid cells containing one of the 119 nodes spanning the EMST. Results are inconclusive. While railroad kilometres inside the buffer around the EMST are strongly associated with areas being too well off today, the same holds to a lesser degree to those outside of it. The estimate for rails close to the EMST is three times as large as the one for rails far away, however both are significantly smaller than zero. This leads to the conclusion that all colonial railways clustered economic activity in inefficient locations, only those built for strategic purposes did so even more. The social planner would like to reorganise infrastructure away from all former railway areas, but she would start at those constructed for geopolitical reasons.

% Figure Table RailBlocks_zeta
\begin{table}[t] \centering
  \caption{General Equilibrium Effects of Colonial Railroads}
  \label{tab:RailBlocks_zeta}
  \resizebox{\textwidth}{!}{


  \begin{tabular}{@{\extracolsep{5pt}}lcccccccc}
  \\[-1.8ex]\hline
  \hline \\[-1.8ex]
   & \multicolumn{8}{c}{\textit{Dependent variable: Local Infrastructure Discrimination Index $\Lambda_{i}$}} \\
   \cline{2-9}
     \\[-1.8ex]
   & \multicolumn{6}{c}{\small Full Sample} & \multicolumn{2}{c}{\small Restricted Sample} \\
  \cline{2-7} \cline{8-9}
  \\[-1.8ex] & (1) & (2) & (3) & (4) & (5) & (6) & (7) & (8)\\
  \hline \\[-1.8ex]
   $<10$ KM to Colonial Railroad & $-$0.013$^{***}$ & $-$0.015$^{***}$ & $-$0.017$^{***}$ &  &  &  & $-$0.014$^{***}$ &  \\
    & (0.003) & (0.004) & (0.004) &  &  &  & (0.004) &  \\
    & & & & & & & & \\
   $10-20$ KM to Colonial Railroad & $-$0.013$^{***}$ & $-$0.016$^{***}$ & $-$0.017$^{***}$ &  &  &  & $-$0.015$^{***}$ &  \\
  & (0.003) & (0.004) & (0.004) &  &  &  & (0.004) &  \\
    & & & & & & & & \\
   $20-30$ KM to Colonial Railroad & $-$0.002 & $-$0.004 & $-$0.005 &  &  &  & $-$0.005 &  \\
  & (0.004) & (0.004) & (0.004) &  &  &  & (0.004) &  \\
    & & & & & & & & \\
   $30-40$ KM to Colonial Railroad & 0.010$^{**}$ & 0.008$^{*}$ & 0.007 &  &  &  & 0.007 &  \\
  & (0.005) & (0.005) & (0.005) &  &  &  & (0.005) &  \\
    & & & & & & & & \\
   $<10$ KM to Colonial Placebo Railroad &  &  &  & $-$0.005 & $-$0.006 & $-$0.006 &  & $-$0.007$^{*}$ \\
  &  &  &  & (0.004) & (0.004) & (0.004) &  & (0.004) \\
    & & & & & & & & \\
   $10-20$ KM to Colonial Placebo Railroad &  &  &  & $-$0.003 & $-$0.004 & $-$0.005 &  & $-$0.005 \\
  &  &  &  & (0.005) & (0.005) & (0.005) &  & (0.005) \\
    & & & & & & & & \\
   $20-30$ KM to Colonial Placebo Railroad &  &  &  & $-$0.001 & $-$0.001 & $-$0.001 &  & $-$0.004 \\
  &  &  &  & (0.004) & (0.004) & (0.004) &  & (0.004) \\
    & & & & & & & & \\
   $30-40$ KM to Colonial Placebo Railroad &  &  &  & 0.007 & 0.006 & 0.005 &  & 0.003 \\
  &  &  &  & (0.004) & (0.004) & (0.004) &  & (0.004) \\
    & & & & & & & & \\
  \hline \\[-1.8ex]
  Country FE & Yes & Yes & Yes & Yes & Yes & Yes & Yes & Yes \\
  Geographic controls &  & Yes & Yes &  & Yes & Yes & Yes & Yes \\
  Simulation controls &  &  & Yes &  &  & Yes & Yes & Yes \\
  Observations & 10,158 & 10,158 & 10,158 & 10,158 & 10,158 & 10,158 & 6,362 & 6,362 \\
  R$^{2}$ & 0.101 & 0.126 & 0.129 & 0.099 & 0.123 & 0.125 & 0.121 & 0.115 \\
  \hline
  \hline \\[-1.8ex]
  \textit{Note:}  & \multicolumn{8}{r}{$^{*}$p$<$0.1; $^{**}$p$<$0.05; $^{***}$p$<$0.01} \\
  \end{tabular}

}

\mysubcaption{This table displays effects of various distance-intervals on the Local Infrastructure Discrimination Index $\Lambda$. Explanatory covariates are dummy-variables indicating whether a cell's centroid is within X kilometres to its closest colonial railroad. Distance larger than 40 kilometres is the omitted category. Geographic controls consist of altitude, temperature, average land suitability, malaria prevalence, yearly growing days, average precipitation, indicators for the 12 predominant agricultural biomes, indicators for whether a cell is within 25 KM of a natural harbor, navigable river, or lake, the fourth-order polynomial of latitude and longitude, and an indicator of whether the grid cell lies on the border of a country's network. Simulation controls are comprised of population, night lights, ruggedness, and a dummy for whether a cell is classified as urban. Columns (1)-(3) examine the effect of actually built colonial railroads. Columns (4)-(6) repeat these calculations with railroads that were planned, but never built (``placebo railroads''). Columns (7)-(8) restrict the sample to the 32 countries on which data for colonial railways is available. Results are robust to excluding all grid cells containing a country's capital (not reported). Heteroskedasticity-robust standard errors are clustered on the 3x3 degree level and are shown in parantheses.}
\end{table}
% End Figure

So far, my empirical analysis has merely revolved around intensive margin effects of colonial railroads. Areas with more railways are stripped off more infrastructure by the social planner. But what about the extensive margin? Until how far away does the confounding effect of former train lines last? Table \eqref{tab:RailBlocks_zeta} displays results from regressing $\Lambda_{i,c}$ on a series of indicators denoting whether a grid cell's centroid is within certain distance intervals from its closest colonial rail line. Columns (1) -- (3) jointly estimate the effects of being within [0 -- 10], (10 -- 20], (20 -- 30], (30 -- 40], or (40+) kilometres from the closest passing railroad (with $(40+)$ being the omitted category). \cite{jedwab_permanent_2016} find these intervals to be most significant in uncovering local reallocation effects. Apart from revealing the geographical extend to which colonial investments persist to skew spatial infrastructure systems today, these analyses can also help uncover general equilibrium effects. Are areas close to railroads better off at the expense of their immediate neighbours?

Estimates from columns (1) -- (3) lend support for this claim. Grid cells with centroids less than 10 kilometres away from a passing railroad line are between $1.3$ and $1.7$ percentage points too well off compared to the omitted category. The estimate is significantly different from zero and robust to gradually introducing additional controls. The same dynamic also holds for cells within 20 kilometres of a passing line, and becomes undetectable further out. Moreover, one can even detect an adverse effect for cells between 30 and 40 kilometres away. Those cells would be granted additional welfare from redistributive efforts by the social planner. Though the estimate becomes increasingly imprecise and even insignificantly different from zero as further controls are introduced, the point estimate does not move much. It seems as if the confounding effect of colonial infrastructure policies is a local one. Areas blessed with a close-by railway line are still too well off today, which comes at the expense of their neighbouring regions just a few kilometres away.

To gain more confidence in the causal nature of this dynamic, columns (4) -- (6) again repeat the same exercise with placebo railroads. Of the twelve estimates produced, none is significantly different from zero. Columns (7) and (8) print results from estimating the same equation with the full set of controls, but restricting the sample to only grid cells in the 34 countries with data on colonial railway investment. In previous estimation, this restriction had virtually no effect on estimates' size, sign, and significance (and hence was not explicitly reported). In the current case, however, truncating the sample by roughly a third produces a barely significant estimate for areas close to placebo railroads. All other results remain more or less the same. While this is certainly noteworthy, I believe this association to be merely spurious. In fact, by the law of large numbers, one would even statistically expect at least one of 16 placebo estimations to be within a p-value of less than $0.1$. The causal nature of local reallocation dynamics initiated by colonial infrastructure investments is hardly attenuated through this adverse finding.

In this section, I have shown how large scale transport infrastructure projects from the colonial period persist to skew the spatial composition of African trade networks towards a suboptimal state. Areas crossed by more colonial railway kilometres are too well off given their position in national trade systems today, even though these railroads were constructed over a century ago. Transport lines devised to meet military or strategic needs are associated with particularly strong effects. I also show that these projects set off local reallocation dynamics through which areas close to a line benefit at the expense of their neighbours. My results contribute to the plethora of studies examining the persistent impact of detrimental colonial policies on current development in Africa. However, recent history is hardly the only cause for heterogeneous development on the continent. In the following sections, I investigate two very different channels through which Africa's trade network might potentially be confounded: ethnic relations and foreign aid.

\subsection{Ethnic Relations}
An ominous legacy of the colonial era is the design of Africa's national borders. Drawn without much regard for local circumstances and with European rather than African interests in mind, many territories amalgamated various previously unrelated tribes and ethnicities under the umbrella of one nationality. After independence, most borders persisted and today African nations are among the most ethnically diverse in the world \citep{Alesina_Ethnicinequality_2016}.

A well established strand of literature has investigated the impacts of ethnic diversity and identity on international comparative development \citep{easterly_africas_1997,Alesina_EthnicDiversityEconomic_2005,gennaioli_modern_2007,Alesina_Ethnicinequality_2016}. More recently, a series of studies has begun to shine light on the causes and effects of ethnicity-level heterogeneity within African countries. Ethnic homelands and tribes have been shown to be better developed when they are represented in national leadership \citep{Franck_DoesLeaderEthnicity_2012}, have more centralised deep rooted institutional systems \citep{michalopoulos_pre-colonial_2013}, or simply are not split in two by artifical national border designs \citep{michalopoulos_long-run_2016}.

The first of these three findings presents a particularly interesting case for the spatial inefficiencies present in my dataset. Are ethnic homelands with less political clout systematically discriminated against in the provision of trade infrastructure? Is the substantial variation in infrastructure discrimination caused by government authorities channeling more transport investments towards members of their own ethnic group, even though other areas are more in need of better roads? Are African trade networks inefficient because of ethnic favouritism? If this were the case, the ethnicity-blind social planner would intervene and shuffle infrastructure away from favoured groups to achieve overall efficiency.

Existing studies have shown that the rise to power of a new national leader leads to temporarily more consumption and output in the leader's birth region \citep{Hodler_RegionalFavoritism_2014} and ethnic homeland \citep{DeLuca_Ethnicfavoritismaxiom_2018}. During the leader's time in office, birth region and ethnic homelands also benefit from more foreign aid being channeled their way \citep{Dreher_AiddemandAfrican_2016}. Most relevant to my research question is a study by \cite{burgess_value_2015}. Using a panel of public road expenditure figures in Kenyan districts over time, they find that areas sharing the ethnicity of the current leader receive almost twice as much infrastructure spending and up to five times as many newly constructed road kilometres than the rest of the country. Remarkably, this ethnic bias attenuates in periods of democratic rule, providing evidence for the efficacy of inclusive political institutions in constraining patronage. Even though the result is robust to a variety of specification tests, it could still be driven by disproportionally higher needs for infrastructure spending in the treated areas. To prove ethnic favouritism, one has to show that not only did leaders channel \emph{more} infrastructure to their coethnic districts, but that these were also \emph{worse} investment opportunities than somewhere else in the country. \citeauthor{burgess_value_2015} address this concern by creating a counterfactual road network said to optimise overall trade flows, which in turn does not reveal ethnic favouritism. However, this counterfactual road system is not based on global network optimisation, but relies on a rather simple heuristic of iteratively connecting areas with high market potential to each other. I will put this finding under scrutiny and investigate whether for all of Africa, ethnicities with more political clout actually have more infrastructure than they optimally should.

% Figure Ethnicity Map
\begin{figure}
\centering
\caption{$\Lambda_{h}$ over Ethnic Homelands}

\includegraphics[width=0.65\textwidth,trim={1cm 4cm 0cm 4.5cm},clip]{/Users/Tilmanski/Documents/UNI/MPhil/Second Year/Thesis_Git/Analysis/output/other_maps/ethnicity_zeta.png}


\label{fig:ethn_maps}
\mysubcaption{Spatial distribution of Local Infrastructure Discrimination Index $\Lambda_{h}$, aggregated over ethnic homelands. Unit of observation is pre-colonial homelands as initially defined in an ethnolinguistic map by \cite{Murdock_Africaitspeoples_1959} intersected by current political borders  \citep[following][]{michalopoulos_long-run_2016}. $\Lambda_{h}$ is from grid cell level and weighted by population. Maps's colouring follows an equal-interval rule such that every colour in the spectrum has an equal amount of members. This is to visualise the measure's variation but leads to unequal bracket-sizes for each colour.}
\end{figure}
% End Figure

To analyse spatial patterns in the Local Infrastructure Discrimination Index $\Lambda$ over ethnic homelands, I follow \cite{michalopoulos_long-run_2016} and intersect an ethnolinguistic map of pre-colonial homelands from \cite{Murdock_Africaitspeoples_1959} with current political borders. That way, ethnicities present in more than one country count as multiple observations. Of the 835 inhabited homelands identified by \citeauthor{Murdock_Africaitspeoples_1959}, 314 are split in two or more parts by the current national borders, creating 1,212 ethnicity-country observations. I spatially merge my grid cell measure of network inefficiency $\Lambda_{i}$ onto this ethnicity-country level by assigning each grid cell an ethnicity based on its centroid location and weighing grid cells by their respective population.\footnote{Recall that on the grid cell level $\Lambda_{i}$ was defined as \begin{equation*}
  \Lambda_{i} = \frac{\textrm{Welfare under the optimal Infrastructure}_{i}}{\textrm{Welfare under the current Infrastructure}_{i}} = \frac{L_{i}u_{i}^{\textrm{optimal Infrastructure}}}{L_{i}u_{i}^{\textrm{current Infrastructure}}}
\end{equation*} To aggregate this onto the ethnic homeland level, I then sum over all grid cells $i$ in homeland $h$ \begin{equation*}
  \Lambda_{h} = \frac{\sum_{i \in h}^{} L_{i}u_{i}^{\textrm{optimal Infrastructure}}}{\sum_{i \in h}^{}L_{i}u_{i}^{\textrm{current Infrastructure}}}
\end{equation*}} 280 ethnic homelands are too small to overlay with any grid cell centroid, leaving me with 932 observations. Figure \eqref{fig:ethn_maps} presents the spatial variation of Local Infrastructure Discrimination $\Lambda_{h}$ on the homeland level. The African ethnicity which would benefit most from national reshuffling of roads is the tiny Tunisian part of the Ghadames homeland, an area identified as an oil basin in the 1990s, which now stands to gain more than 40\% of welfare and hence arguably presents an outlier \citep{Echikh_Geologyhydrocarbonoccurrences_1998}. Other ethnicities which are discriminated against mostly by the current network are the Kababish in Sudan (25\%), the Bata in Cameroon (24\%), or the Aushi in Congo-Kinshasa (23\%). The most disproportionally advantaged ethnic homelands in Africa are those of the Mober in Nigeria (which stand to lose 8\% of welfare if optimal networks were imposed), the Sanga (7\%), and Kreish (11\%) both in the Central African Republic.

Using \citeauthor{Murdock_Africaitspeoples_1959}'s map on pre-colonial ethnolinguistic characteristics to identify current locations of ethnic homelands is not unproblematic. Firstly, the map originally appeared in small print in the appendix of \citeauthor{Murdock_Africaitspeoples_1959}'s book and was only later vectorised. When digitally scaling it up, the borders between homelands will naturally be subject to imprecision \citep{michalopoulos_pre-colonial_2013}. This should not be of much concern for the present study, as my underlying dataset on network inefficiency itself already only comes at a resolution of roughly 55 by 55 kilometres. The imprecise drawing of an otherwise detailed map hence arguably does not add much noise to my already noisy dataset (bearing in mind that the smallest homelands, which are most likely to be affected by drawing imprecision are automatically dropped from my dataset by way of constructing $\Lambda_{h}$, see above). Secondly, as \cite{michalopoulos_pre-colonial_2013} point out, a map from 1959, reflecting homelands from the pre-colonial era, might not be an accurate representation of the current ethnic topography. It is reasonable to assume that people will have migrated away from their original ethnic homeland in a way that is endogenous to many of the (mainly development) variables I consider in this study. \cite{nunn_slave_2011}, however, find that a majority of the continent's population still live in their ancestors' ethnic homelands, partially assuaging this concern. It is also worth emphasising that I use \citeauthor{Murdock_Africaitspeoples_1959}'s map merely as a level of aggregation in order to gauge ethnic discrimination. Even if substantial migration patterns had taken place, it is still worthwhile to investigate whether governments treat homelands differentially based on their historic ethnic attribution.\footnote{Other than \cite{michalopoulos_pre-colonial_2013,michalopoulos_long-run_2016}, I also do not make use of any of the auxiliary ethnolinguistic data provided in \citeauthor{Murdock_Africaitspeoples_1959}'s book. The definition of ethnic homelands is exclusively used to obtain an approriate level of aggregation, not to characterise any economic, societal, or institutional differences between homelands.} A third concern is that homelands are coded as non-overlapping, mutually exclusive regions. In this conception, there are no diverse areas with multiple coexisting ethnicities. While this might be accurate for most rural areas, it is an untenably strong assumption for big, primal cities. Especially the capital city naturally attracts migrants from all parts of a country (and beyond), regardless of its immediately surrounding homeland. To counter this challenge, I follow \cite{Alesina_Ethnicinequality_2016} and exclude all ethnic homelands containing a country's capital city as a robustness check. As discussed below, this mostly does not alter results.

I investigate two patterns of infrastructure discrimination along ethnic lines, namely the preferential treatment of  groups sharing the ethnic homeland with the national leaders (\emph{ethnic favoritism}) and the negative treatment of groups excluded from the government (\emph{ethnic discrimination}). To measure the latter, I rely on four measures of political relationships between ethnicities. Firstly, I make use of the Ethnic Power Relations (EPR) database by \cite{Vogt_IntegratingDataEthnicity_2015}, which globally identifies ``politically relevant ethnic groups and their access to state power'' over the last seven decades. Not every ethnic homeland inhabits a group that is ``politically relevant'', significantly truncating the sample by 46\%.\footnote{Merging EPR observations with ethnic homelands is non-trivial. Thankfully, I am able to rely on the conversion established by \cite{michalopoulos_long-run_2016}.} EPR reports a yearly time series of political discrimination for every group in their sample. In particular, a group is coded as discriminated against by the central government if there is ``active, intentional, and targeted discrimination by the state against group members in the domain of public politics'' \citep[p.1331]{Vogt_IntegratingDataEthnicity_2015}. As my dataset comes in the form of a simple cross-section without a time dimension, I follow \cite{michalopoulos_long-run_2016} and analyse effects of a dummy variable taking on the value one if a group has experienced discrimination in at least one year between 1960 and 2010. I use this measure to investigate whether infrastructure discrimination covaries with political discrimination. Do groups which are actively discriminated in their political participation also see less than optimal trade investments in their homelands? Secondly, I broaden the definition of ethnic discrimination and more generally look at groups which are excluded from the central government. As defined by EPS, this classification entails all groups that are discriminated against (from above), plus groups that are defined as either powerless or self-excluded \citep[p.1331]{Vogt_IntegratingDataEthnicity_2015}.\footnote{I again rely on the transformation by \cite{michalopoulos_long-run_2016} who code the indicator as one if the ethnic group has experienced exclusion from the government at any point between 1960 and 2010.} Broadening the conception of ethnic discrimination allows for a more nuanced analysis of how non-access to power can influence network inefficiency. Thirdly, I analyse the effects of an indicator denoting whether an ethnicity was part of a civil war with an explicitly ethnic dimension at some point between 1960 and 2010. The construction of this indicator is identical to the ethnic discrimination dummy described above and is also taken from \cite{michalopoulos_long-run_2016}. As a more indirect measure for ethnic discrimination, I finally analyse whether ethnicities which were split in two or more parts by the arbitrary border drawings of the colonial powers get less than optimal infrastructure investment. \cite{michalopoulos_long-run_2016} show that ethnic homelands of which ten or more per cent are in more than one country are significantly more prone to violence, have less political clout, and report less well-being.

I also analyse the reverse effects of ethnic favoritism. Are ethnicities systematically better off when the country's leader was born in their homeland? To answer this, I make use of a dataset of 117 African national leaders provided by \cite{Dreher_AiddemandAfrican_2016}. The data entails information about the birth region and time in office of heads of state of holding power in 44 African countries dating back to 1969.\footnote{No data on national leaders is reported for Algeria, Western Sahara, South Sudan, Somalia, and Djibouti. Even for countries with data, coverage is not comprehensive as the dataset excludes leaders born abroad or with unknown birthplaces. 93.6\% of homelands never sent anyone to the highest office in the country. For those that did, tenures last from merely one year (the Zerma in Niger) to 42 years (the Duma in Gabon, homeland of long-term head of state Omar Bongo).} Using Open Street Maps, I obtain coordinates for birthplaces and spatially merge them with my dataset on ethnic homelands. I then use this information to calculate for each ethnic group the total number of years someone born in the respective homeland has held high office. This allows me to analyse whether a homeland's over provision with transport infrastructure covaries with personal ties to national power.

To investigate patterns of infrastructure discrimination on the ethnic homeland level, I estimate a slightly different version of Equation \eqref{eq:grid_ols}
\begin{equation}
  \Lambda_{h,c} = \beta v_{h,c} + \textbf{X}_{h,c}\gamma + \delta_{c} + \epsilon_{h,c}
  \label{eq:ethn_ols}
\end{equation}
where $\Lambda_{h,c}$ is the Local Infrastructure Discrimination Index for homeland $h$ in country $c$, $\textbf{X}_{h,c}$ and $\delta_{c}$ again denote controls and country-fixed effects respectively, $v_{h,c}$ are the explanatory covariates discussed above, and $\beta$ is the coefficient of interest. The number of ethnicity observations (about 900) is significantly smaller than the number of grid cells observations employed so far. In order to avoid overfitting, I slightly truncate the set of controls $\textbf{X}_{h,c}$ and replace the latitude and longitude polynomials, the classification into urban or rural, as well as dummies indicating proximity to a natural harbor, river, lake, and national border with two continuous measures of distance to the nearest border and distance to the coast. As homelands are much more irregularly shaped than grid cells, I also include the logarithm of each homelands' area \citep{michalopoulos_long-run_2016}. Other than these adjustments, $\textbf{X}_{h,c}$ entails all geographical and simulation controls as in the models on the grid cell level. Finally, as many ethnicities appear more than once, the error term will reasonably be autocorrelated beyond the country-level. To account for this, I follow \cite{michalopoulos_long-run_2016} and double-cluster standard errors at both the country level as well as the ethnic family level using the mechanism proposed by \cite{Cameron_RobustInferenceMultiway_2011}.

% Figure Table Ethnic Discrimination
\begin{table}[t] \centering
  \caption{Null Effect of Ethnic Discrimination}
  \label{tab:Ethn_discrimination}
  \resizebox{\textwidth}{!}{


  \begin{tabular}{@{\extracolsep{5pt}}lcccccccc}
  \\[-1.8ex]\hline
  \hline \\[-1.8ex]
   & \multicolumn{8}{c}{\textit{Dependent variable:}} \\
  \cline{2-9}
  \\[-1.8ex] & (1) & (2) & (3) & (4) & (5) & (6) & (7) & (8)\\
  \hline \\[-1.8ex]
   Ethnicity discriminated against 1960--2010 & $-$0.001 & $-$0.001 &  &  &  &  &  &  \\
    \hspace*{3mm} \small (SE) & (0.008) & (0.007) &  &  &  &  &  &  \\
    \hspace*{3mm} \footnotesize \textit{[MDE]} & \textit{\footnotesize [0.021]} & \textit{\footnotesize [0.021]} &  &  &  &  &  &  \\
    & & & & & & & & \\
   Ethnicity excluded from the &  &  & $-$0.006 & $-$0.005 &  &  &  &  \\
    \hspace*{3mm} central government 1960--2010 &  &  & (0.005) & (0.005) &  &  &  &  \\
    &  &  & \textit{\footnotesize [0.014]} & \textit{\footnotesize [0.014]} &  &  &  &  \\
    & & & & & & & & \\
   Ethnicity involved in an &  &  &  &  & 0.002 & 0.002 &  &  \\
    \hspace*{3mm}  ethnic war 1960--2010 &  &  &  &  & (0.008) & (0.008) &  &  \\
    &  &  &  &  & \textit{\footnotesize [0.022]} & \textit{\footnotesize [0.022]} &  &  \\
    & & & & & & & & \\
   Ethnicity split by colonial borders &  &  &  &  &  &  & $-$0.002 & $-$0.002 \\
   &  &  &  &  &  &  & (0.004) & (0.004) \\
    &  &  &  &  &  &  & \textit{\footnotesize [0.011]} & \textit{\footnotesize [0.012]} \\
    & & & & & & & & \\
  \hline \\[-1.8ex]
  Country FE & Yes & Yes & Yes & Yes & Yes & Yes & Yes & Yes \\
  Geographic controls & Yes & Yes & Yes & Yes & Yes & Yes & Yes & Yes \\
  Simulation controls &  & Yes &  & Yes &  & Yes &  & Yes \\
  Observations & 496 & 496 & 496 & 496 & 496 & 496 & 932 & 932 \\
  R$^{2}$ & 0.156 & 0.166 & 0.158 & 0.168 & 0.156 & 0.167 & 0.164 & 0.167 \\
  \hline
  \hline \\[-1.8ex]
  \textit{Note:}  & \multicolumn{8}{r}{$^{*}$p$<$0.1; $^{**}$p$<$0.05; $^{***}$p$<$0.01} \\
  \end{tabular}

}

\mysubcaption{Statistically insignificant effects of various indicators of ethnic discrimination on the local infrastructure discrimination index. The sample comprises homelands of ethnic homelands, projected on current national borders. Independent variable in columns (1) -- (2) is a dummy variable indicating if an ethnicity has experienced discrimination from the government at some point between 1960-2010. In columns (3) -- (4), independent variable is a dummy indicating if an ethnicity has been excluded from the central government at some point between 1960-2010. (5) -- (6) analyse impacts of ethnicities being involved in a major ethnic war at some point between 1960--2010. All data gathered by \cite{Vogt_IntegratingDataEthnicity_2015} and obtained from \cite{michalopoulos_long-run_2016}. Explanatory variable in (7) and (8) is an indicator of ethnicities being split by national borders, defined as having at least 10 per cent of their homeland in more than one country \citep[from][]{michalopoulos_long-run_2016}. Data for regressions in  (1) -- (6) only exist for politically relevant ethnic groups, truncating the sample by 46\%. All observations exclude Western Sahara, for which no ethnic homeland data exist. Geographic controls consist altitude, temperature, average land suitability, malaria prevalence, yearly growing days, average precipitation, indicators for the 12 predominant agricultural biomes, distances to the nearest coast and border, and the natural logarithm of the homeland area.  Simulation controls are comprised of population, night lights, and ruggedness. Results are robust to excluding homelands containing a country's capital (not reported). Heteroskedasticity-robust standard errors are double-clustered on the country level and the ethnic-family level and are reported in parantheses. All columns also report minimum detectable effect sizes (MDE) in brackets. This is the smallest effect that would have still been detectable with 80\% power at 5\% significance \citep{Haushofer_ShorttermImpactUnconditional_2016}.}
\end{table}
% End Figure

Table \eqref{tab:Ethn_discrimination} reports results of estimating Equation \eqref{eq:ethn_ols} for each of the four indicators of ethnic discrimination. As before, the table prints estimates both with and without the set of simulation controls (consisting of population, night lights, and ruggedness). No estimate is significantly different from zero, implying that the null hypothesis of no linear relation between $v_{h,c}$ and $\Lambda_{h,c}$ cannot be rejected. Inefficiency of the national trade network does not systematically covary with ethnicities historically discriminated against (columns 1--2), excluded from the government (3--4), involved in an ethnic war (5--6), or split by arbitrary colonial borders (7--8). Together , these results do not support the contention that discriminated ethnic groups are systematically disadvantaged in the design of national trade networks. After the social planner has reshuffled a country's roads, historically victimised groups are not better off than they were before. These (null) findings do not change when (a) excluding homelands containing a nation's capital, (b) including ethnic family fixed effects (c) controlling for pre-colonial differences in societal structure between ethnicities, namely complexity of hierarchies, the existence of compact settlement structures, and existence of a class system \citep{michalopoulos_pre-colonial_2013}, (d) using a more lenient definition of split homelands whereby only 5\% of the territory has to be on more than one country (instead of 10\%), (e) clustering standard errors along merely (either) one dimension, and (f) excluding all geographic controls. All estimates hardly move and remain statistically indistinguishable from zero (not reported).

With null results, it can be a challenge to attribute the finding to there actually not being any underlying relationship in the data, or the study merely being underpowered to detect a reasonably sizeable effect. To gain an understanding for which of the two explanations is likely to account for the results at hand, I follow a recent contribution by \cite{Haushofer_ShorttermImpactUnconditional_2016} and calculate the \emph{Minimum Detectable Effect Size} (MDE) for each estimate. This is the smallest estimate that would have still been detectable with 80\% power at the 5\% significance level. As \cite{Haushofer_ShorttermImpactUnconditional_2016} demonstrate, the MDE can be computed as
\begin{equation*}
  \textrm{MDE} = (t_{1-\kappa}+t_{0.5\alpha}) \times SE(\hat{\beta})
\end{equation*}
where $SE(\hat{\beta})$ is the standard error of the estimated coefficient, and $t_{1-\kappa}=0.84$ and $t_{0.5\alpha}=1.96$ denote the values of the t-statistic required to achieve 80\% power and 5\% significance, respectively. This trivially simplifies to $\textrm{MDE} = 2.8 \times SE(\hat{\beta})$. I use this measure to gauge which order of magnitude my statistical tests are generally capable of detecting. If the MDE is very large, the underlying impact on $\Lambda$ would have to be substantial in order to even be noticed. Table \eqref{tab:Ethn_discrimination} displays MDEs in brackets under each estimate's standard error. Effects of ethnic discrimination and involvement in an ethnic war (columns 1--2 and 5--6) are slightly larger than $0.02$. This implies that victimising an ethnicity would have to lead to at least a 2\% increase (or reduction) in Local Infrastructure Discrimination $\Lambda_{h}$ -- slightly larger than being very close to a colonial railway (see Table \ref{tab:RailBlocks_zeta}). Considering the importance of ethnic power relations on subregional development in Africa, I consider this a reasonably fine resolution. Minimum detectable effect sizes for ethnicities excluded from the government (columns 3--4) or split by colonial borders (7--8) are even smaller and hence powered to detect even smaller effects. This leads to the conclusion that the reported null effects of Table \eqref{tab:Ethn_discrimination} are not just an artificat of underpowered tests, but rather bolster the contention that active ethnic victimisation and trade network inefficiency are not systematically linked.

% Figure Table favoritism
\begin{table}[!t] \centering
  \caption{Ethnic and Regional Favoritism}
  \label{tab:favoritism}
  \resizebox{\textwidth}{!}{


  \begin{tabular}{@{\extracolsep{5pt}}lcccccccc}
  \\[-1.8ex]\hline
  \hline \\[-1.8ex]
   & \multicolumn{8}{c}{\textit{Dependent variable: Local Infrastructure Discrimination Index $\Lambda$}} \\
  \cline{2-9} \\[-1.8ex]
& \multicolumn{5}{c}{Full Sample} & \multicolumn{3}{c}{Excluding Capitals} \\
  \cline{2-6}   \cline{7-9}
  \\[-1.8ex] & (1) & (2) & (3) & (4) & (5) & (6) & (7) & (8)\\
  \hline \\[-1.8ex]
  \multicolumn{9}{l}{\textit{Panel A: Ethnicities}} \\
   Years in Power & $-$0.0004$^{**}$ & $-$0.0004$^{**}$ & $-$0.001$^{**}$ &  &  & $-$0.0003$^{*}$ & $-$0.0005$^{***}$ &  \\
  & (0.0002) & (0.0002) & (0.0003) &  &  & (0.0002) & (0.0002) &  \\
    & & & & & & & & \\
   Years in Power $\times$ Democracy  &  &  & 0.0004 &  &  &  & 0.0003 &  \\
  &  &  & (0.0003) &  &  &  & (0.0003) &  \\
    & & & & & & & & \\
   In Power Dummy &  &  &  & $-$0.007$^{*}$ & $-$0.008$^{**}$ &  &  & $-$0.007$^{*}$ \\
  &  &  &  & (0.004) & (0.004) &  &  & (0.004) \\
    & & & & & & & & \\
    Country FE & Yes & Yes & Yes & Yes & Yes & Yes & Yes & Yes \\
    Geographic controls & Yes & Yes & Yes & Yes & Yes & Yes & Yes & Yes \\
    Simulation controls &  & Yes & Yes &  & Yes & Yes & Yes & Yes \\
    Observations & 932 & 932 & 932 & 932 & 932 & 895 & 895 & 895 \\
    R$^{2}$ & 0.165 & 0.169 & 0.169 & 0.165 & 0.169 & 0.177 & 0.177 & 0.177 \\
   \hline \\[-1.8ex]
\multicolumn{9}{l}{\textit{Panel B: Grid Cells}} \\
  Years in Power & $-$0.001$^{***}$ & $-$0.001$^{***}$ & $-$0.001$^{***}$ &  &  & $-$0.001$^{***}$ & $-$0.001$^{**}$ &  \\
   & (0.0003) & (0.0002) & (0.0004) &  &  & (0.0003) & (0.0004) &  \\
   & & & & & & & & \\
  Years in Power $\times$ Democracy &  &  & $-$0.0001 &  &  &  & $-$0.0002 &  \\
   &  &  & (0.001) &  &  &  & (0.001) &  \\
   & & & & & & & & \\
  In Power Dummy &  &  &  & $-$0.024$^{***}$ & $-$0.025$^{***}$ &  &  & $-$0.026$^{***}$ \\
   &  &  &  & (0.006) & (0.006) &  &  & (0.007) \\
   & & & & & & & & \\

 Country FE & Yes & Yes & Yes & Yes & Yes & Yes & Yes & Yes \\
 Geographic controls & Yes & Yes & Yes & Yes & Yes & Yes & Yes & Yes \\
 Simulation controls &  & Yes & Yes &  & Yes & Yes & Yes & Yes \\
 Observations & 10,066 & 10,066 & 10,066 & 10,066 & 10,066 & 10,019 & 10,019 & 10,019 \\
 R$^{2}$ & 0.124 & 0.125 & 0.125 & 0.124 & 0.126 & 0.128 & 0.128 & 0.128 \\
 \hline
 \hline \\[-1.8ex]
 \textit{Note:}  & \multicolumn{8}{r}{$^{*}$p$<$0.1; $^{**}$p$<$0.05; $^{***}$p$<$0.01} \\
 \end{tabular}

}

\mysubcaption{Persistent impacts of holding power on Local Infrastructure Discrimination Index $\Lambda$ in leaders' homelands and birthplaces. Panel A estimate on the sample of ethnic homelands split by current national borders. Independent variable in columns (1)--(3) is the number of years since 1969 someone born in the homeland was the country's leader. Column (3) adds an interaction with a dummy indicating if the homeland is in a democratic country. In (4) -- (5), independent variable is a dummy indicating whether the homeland ever held power. Columns (6) - (8) replicate the regressions while excluding observations containing a country's capital. Panel B estimate similar models on the grid cell level. Data for leaders' birthplaces from \cite{Dreher_AiddemandAfrican_2016}, countries' democracy classification from \cite{Marshall_PolityProjectCenter_2015}. All observations exclude Western Sahara, for which no ethnic homeland data exist. Geographic controls consist of altitude, temperature, average land suitability, malaria prevalence, yearly growing days, average precipitation, indicators for the 12 predominant agricultural biomes. For ethnic homelands also distances to the nearest coast and border, and the natural logarithm of the homeland area. For grid cells also indicators for whether a cell is within 25 KM of a natural harbor, navigable river, or lake, the fourth-order polynomial of latitude and longitude, and an indicator of whether the grid cell lies on the border of a country's network. Simulation controls are comprised of population, night lights, ruggedness, and whether a cell is classified urban (for grid cells). Heteroskedasticity-robust standard errors are (double-)clustered on the country level (and the ethnic-family level for homelands) and are reported in parantheses.}
\end{table}
% End Figure

Table \eqref{tab:favoritism} investigates the reverse effect of ethnic and regional favouritism. Panel A, columns (1) through (5) estimate Equation \eqref{eq:ethn_ols} on the full sample of ethnic homelands intersected with current national borders. In (1) and (2), the explanatory variable is the total number of years someone born in the ethnic homeland was holding national power. The covariate enters with a significant and negative coefficient, implying that for each year one of their members was in power, an ethnic homeland is about 0.04\% too well off given their relative position in the country's trade network. If the social planner were to intervene, she would strip the homeland with ethnic ties to power from some infrastructure and reallocate them towards areas with no such ties. The effect size is small -- with the median leader staying in power for nine years, ethnic favouritism distorts network efficiency in the order of magnitude of less than half a percentage point. Nevertheless, if provides further evidence for many of the recent findings on ethnic favouritism \citep{DeLuca_Ethnicfavoritismaxiom_2018}.

\cite{DeLuca_Ethnicfavoritismaxiom_2018}, \cite{Hodler_RegionalFavoritism_2014}, and \cite{burgess_value_2015} have hypothesised that ethnic favouritsm is less prevalent in more democratic countries, especially on the African continent. To test for similar patterns in my setting, I follow \citeauthor{DeLuca_Ethnicfavoritismaxiom_2018} and interact total years in power with a variable capturing democratic institutional quality, as reported by \cite{Marshall_PolityProjectCenter_2015}. The variable comes from the Polity4 project and ranks each country on a scale from -10 (authoritarian) to 10 (democratic). I transform this score to a dummy equalling one if a country had a positive rating in 2016. Column (3) reports results of estimating equation \eqref{eq:ethn_ols} including an interaction term.\footnote{The equation to be estimated is \begin{equation*}
  \Lambda_{h,c} = \beta_{1} YearsInPower_{h,c} + \beta_{2} YearsInPower_{h,c}\times Democracy_{h,c} + \textbf{X}_{h,c}\gamma + \delta_{c} + \epsilon_{h,c}
\end{equation*} with the country-fixed effect capturing any ex-ante difference between countries' score in $Democracy_{h,c}$.} The coefficient for the number of years in power is strongly attenuated, yet remains significant at the 5\%. More importantly, the interaction term does not enter with an estimate significantly different from zero. Countries that were democratic in 2016 are not performing better or worse in curtailing ethnic favouritism. After having analysed intensive margin effects of the number of years in power, columns (4) and (5) extend the inquiry to extensive margin effects. The explanatory variable is now a dummy equalling one if an ethnic homeland has ever had someone represent them as head of state (regardless of how long). The coefficient enters significantly and negatively, implying that ethnic favouritism is not constrained to a few long-term national leaders.

As will be discussed below, all regressions of Table \eqref{tab:favoritism} are not perfectly identified. While I do control for an extensive set of auxiliary variables, unobservable differences between homelands might still simultaneously impact both their infrastructure provision, and their chances of sending some of their own to be national leader. This bias is particularly evident for national capitals: leaders disproportionally are born in the capital, and capitals also have significantly better infrastructure provision. To try to account for this confound, I re-estimate columns (2), (3), and (5) while excluding all homelands which include a nation's capital. Results are printed in columns (6) -- (8) and are not qualitatively different from the full sample estimates. Significance becomes slightly weaker in (6) and (8), yet slightly stronger in (7). The effect thus prevails even in homelands not immediately connected to the places of power.

Regional favouritism, the notion that a leader would disproportionally channel government resources towards his or her birthplace (rather than ethnic group), has been a competing hypothesis to favouritism on ethnic grounds. I explore this potential source of network inefficiency in Panel B of Table \eqref{tab:favoritism}. Again using the dataset by \cite{Dreher_AiddemandAfrican_2016}, I identify birthplaces of national leaders and assign them to individual grid cells. The 10,000+ cells are on average much smaller than ethnic homelands and hence allow to detect preferential treatment on a much more confined regional scale. The exposition of columns (1) through (8) immediately follows Panel A, yet the set of controls is slightly altered to match the grid cell level strategy of the rest of this study.\footnote{As with all regressions on the grid cell level, Equation \eqref{eq:grid_ols} is estimated, full geographic and simulation controls are included, and standard errors are clustered solely along the country dimension.} Results are similar to Panel A, yet much stronger. Per year someone born in a given grid cell held the country's highest office, the cell has 0.1 percentage points too much welfare compared to the social optimum (columns 1--2). The premium of ever producing a national leader is estimated at 2.5 percentage points (column 5) -- an effect equivalent to the difference between a cell without any colonial railroads and the cell with the highest total amount of railway kilometres in Africa. These estimates do not change when allowing for differentiation by democratic institutions (column 3) or excluding all grid cells containing a capital (6--8). Taken at face value, the results strongly suggest the presence of regional favouritism distorting the spatial optimum of African trade networks.

As alluded to above, the estimations of Table \eqref{tab:favoritism} are notoriously afflicted with endogeneity concerns. Even after controlling for many immediate confounds like population, economic activity, geographical characteristics, or country fixed effects, one cannot rule out that areas producing a national leader are still substantially different along unobservables. Hard-to-quantify idiosyncracies like an area being a historical party stronghold or having been of strategic importance in a military coup which produced a new head of state might lead to disproportionally many leaders from a certain region, while also explaining differential infrastructure investments into the area. Most of the well-identified studies on ethnic or regional favouritism solve this problem by introducing a time-dimension. \cite{DeLuca_Ethnicfavoritismaxiom_2018}, \cite{burgess_value_2015}, and \cite{Hodler_RegionalFavoritism_2014} all fit panel equations with region fixed effects, accounting for any ex-ante differences between areas. They hence isolate the effect of preferential treatment \emph{while in office}. Since my dataset comes as a mere cross-section, I can only investigate effects of favouritism which might have occurred \emph{at any time in the past}.\footnote{There have been studies attempting to isolate the relationship without time dimension. \cite{Soumahoro_LeadershipfavouritismAfrica_2015} regresses current night lights on past leadership tenures without a panel and finds a very strong effect. These results have been deemed vastly too large by the subsequent literature which had the luxury of including time fixed effects. I interpret this as further evidence for heterogeneity along unobservables confounding a simple cross section, which could reasonably also affect my results.} While this clearly dilutes the robustness of my findings, it is worth bearing in mind that my outcome variable also has a quite different interpretation. While most existing studies analyse variations in flow variables like annual night lights or government expenditures, my measure of network inefficiency is a stock variable. As infrastructure is highly persistent, $\Lambda$ encapsulates information about innumerable transport investment decisions made in the past. If a leader was in power for only a short period in the 1990s, but used the time in office to heavily cater to his or her birth region, one would still expect to see traces of this behaviour in the current infrastructure discrimination index. Regressing current network outcomes on past leadership tenures is at least in principle justified. However, as I cannot rule out the confounding power of unobservables, one should take the results of Table \eqref{tab:favoritism} with a grain of salt.\footnote{One can try to quantify the distorting power unobservables would need to have in order to wash away the entire effect using the technique proposed by \cite{Altonji_SelectionObservedUnobserved_2005}. Using the result without simulation controls in column (1), Panel B, Table \eqref{tab:favoritism} as restricted sample estimate $\beta^{R}$ and the one with simulation controls (column 2) as full sample estimate $\beta^{F}$, I can gauge how many more times I would have to introduce a control set equally powerful in order to attenuate all of the detected relationship. Using the derivation by \cite{nunn_slave_2011}, this is easily computed as $\beta^{F}/(\beta^{R}-\beta^{F})$. With the given estimates in Table \eqref{tab:favoritism}, I obtain a measure of 124.9. In other words, I would have to be unaware of unobservables 125 times as powerful as the additional controls added in column (2) -- night lights, population, urban characteristics, and ruggedness -- in order to falsely detect an effect where there is none. As these are fundamental determinants of economic activity, it is unlikely that unobservables of similar magnitude still exist. While this might sound promising, it is worth noting that the literature has recently warned against relying too much on such tests, especially in environments with much unexplained variation and low $R^{2}$ values \citep[see][]{Oster_UnobservableSelectionCoefficient_2018}.}

Ethnic and regional favouritism are not mutually exclusive. A leader can channel government resources towards inefficient infrastructure projects in both his or her ethnic homeland, as well as distinct birth region. Indeed, by way of constructing the indicators, a leader's birthplace is always contained within his or her ethnic homeland, making it hard to distinguish between the two effects. As a first attempt, I have also constructed a measure of network inefficiency on the homeland level \emph{excluding} the one grid cell in which a leader was born. I obtain this by cutting each rectangular birthplace cell out of the much bigger homeland areas and aggregating over this newly constructed polygon, as proposed by \cite{DeLuca_Ethnicfavoritismaxiom_2018}. When replicating Panel A of Table \eqref{tab:favoritism} with this newly created measure as dependent variable, all estimates turn insignificant (not reported). While this might naively be interpreted as evidence that favouritism does not extend beyond small birth regions, one has to again be cautious of censoring the dependent variable in such a way. If birth regions are inherently different, cutting them out biases estimates on the remaining sample. With the existing data, I am unable to clearly attribute the effects to one of the two explanations.

In this section, I have investigated how ethnic relations skew trade networks towards a sub-optimal state. While I have not found any evidence suggesting that vulnerable and victimised groups are systematically disadvantaged by the current trade system, I can descriptively show that the reverse is true -- ethnicities and regions which have one of their members hold national power are systematically too well off given their relative position in the network. I have presented evidence that this premium exists for both large ethnic homelands and locally confined birth places. Further research is needed to clearly distinguish between the two.

\subsection{Foreign Aid}
Africa is the primary target of international aid. In 2017, no other world region was awarded more fiscal development disbursements from the World Bank -- indeed, Africa received more aid than Europe, Central Asia, Latin America, and the Caribbean combined \citep{TheWorldBank_WorldBankAnnual_2017}. Of these almost 12 billion US dollars worth of lending commitments, by far the biggest share was awarded for projects aimed at improving transportation infrastructure.\footnote{The transport sector made up 18\% of total IBRD and IDA lending to African nations, followed by water and sanitation (14\%), energy and extractives (14\%), and public administration (12\%) \citep{TheWorldBank_WorldBankAnnual_2017}.} The World Bank is not alone -- in the past decade, non-traditional players have entered and disrupted the international development aid system \citep{Dreher_Rogueaidempirical_2015}. Most notably, China has emerged as a significant donor nation, funding development projects in at least 50 African countries totalling more 73 billion US dollars since the turn of the millennium.\citep{Strange_TrackingUnderreportedFinancial_2017}.

Despite the vast amount of resources involved, foreign aid has not yet been unequivocally proven to be linked with positive economic outcomes in recipient countries. An influential literature has long debated whether aid leads to economic growth on the country level \citep{Burnside_AidPoliciesGrowth_2000,Easterly_AidPoliciesGrowth_2004,Rajan_Aidgrowthwhat_2008}. Plagued with data shortcomings and identification challenges, the discourse remained inconclusive for years, up to a point where frustrated researchers were even calling into question the very merit of the entire research agenda \citep{Clemens_CountingChickenswhen_2012,Clemens_NewRoleWorld_2016}. New advances in geo-referenced data availability \citep{Dreher_Aidgrowthregional_2015,Dreher_AidChinaGrowth_2017} and instrumentation techniques \citep{Nunn_USFoodAid_2014,Clemens_CountingChickenswhen_2012}, however, have reignited the literature on the effects of foreign aid.

How do development aid projects relate to my measure of inefficient trade networks? The United Nations have summarised international aid efforts as 'leaving no one behind' \citep{Briggs_LeavingNoOne_2018} and the World Bank is pledging to eradicate extreme poverty \citep{Clemens_NewRoleWorld_2016}. One could therefore expect international aid projects to disproportionally target areas that are discriminated against -- places which have less than they deserve. For transport projects in particular, international donor organisations could almost play a role similar to the social planner in my infrastructure reallocation scenario.

% Contributions to this literature have notoriously been afflicted both with data shortcomings as well as identification challenges. Firstly, as \cite{Clemens_NewRoleWorld_2016} argue, investigating the link between international aid flows and macroeconomic developments is misguided as most development aid is spent to promote relatively small-scale, regional projects. While these might ultimately contribute to growth on the national level, one should not expect such local interventions to immediately propagate into large, macroeconomic turnarounds.\footnote{Additionally, \cite{Clemens_NewRoleWorld_2016} argue that much development aid is targeted at eradicating extreme poverty, not at igniting economic growth. Judging the efficacy of an aid lending line by its impact on national economic growth is hence severely misguided, the authors conclude.} To accurately determine the impact of foreign aid projects, a much finer resolution of their regional distribution is needed. Recently, a series of new datasets have made such analyses possible. With geo-referenced information on the exact location of aid projects, \cite{Dreher_Aidgrowthregional_2015} investigate the relationship between between World Bank development and regional economic growth (and find no causal effect). \cite{Dreher_AidChinaGrowth_2017} conduct a similar analysis with geo-referenced Chinese aid projects in Africa.

% Figure Table Worldbank
\begin{table}[t] \centering
  \caption{Foreign Aid Projects}
  \label{tab:WB_aid}
  \resizebox{\textwidth}{!}{


  \begin{tabular}{@{\extracolsep{5pt}}lcccccccc}
  \\[-1.8ex]\hline
  \hline \\[-1.8ex]
   & \multicolumn{8}{c}{\textit{Dependent variable: Local Infrastructure Discrimination Index $\Lambda$}} \\
  \cline{2-9} \\[-1.8ex]
  \\[-1.8ex] & (1) & (2) & (3) & (4) & (5) & (6) & (7) & (8)\\
  \hline \\[-1.8ex]
  \multicolumn{9}{l}{\textit{Panel A: Worldbank Projects}} \\
  \\[-1.8ex]
  Total aid flows & $-$0.0002$^{***}$ & $-$0.0003$^{***}$ &  &  &  &  &  &  \\
  \hspace*{3mm} in million USD & (0.00004) & (0.00005) &  &  &  &  &  &  \\
   & & & & & & & & \\
  Aid flows for transport projects &  &  & $-$0.0005$^{***}$ & $-$0.001$^{***}$ &  &  &  &  \\
  \hspace*{3mm} in million USD &  &  & (0.0001) & (0.0002) &  &  &  &  \\
   & & & & & & & & \\
  Total number of aid projects &  &  &  &  & $-$0.001$^{***}$ & $-$0.002$^{***}$ &  &  \\
   &  &  &  &  & (0.0002) & (0.0002) &  &  \\
   & & & & & & & & \\
  Number of transport projects &  &  &  &  &  &  & $-$0.002$^{***}$ & $-$0.003$^{***}$ \\
   &  &  &  &  &  &  & (0.0004) & (0.001) \\
   & & & & & & & & \\
 Country FE & Yes & Yes & Yes & Yes & Yes & Yes & Yes & Yes \\
 Geographic controls & Yes & Yes & Yes & Yes & Yes & Yes & Yes & Yes \\
 Simulation controls &  & Yes &  & Yes &  & Yes &  & Yes \\
 Observations & 10,158 & 10,158 & 10,158 & 10,158 & 10,158 & 10,158 & 10,158 & 10,158 \\
 R$^{2}$ & 0.125 & 0.127 & 0.125 & 0.127 & 0.127 & 0.132 & 0.127 & 0.131 \\
   \hline \\[-1.8ex]

\multicolumn{9}{l}{\textit{Panel B: Chinese Development Projects}} \\
\\[-1.8ex]
Total aid flows & $-$0.00004$^{***}$ & $-$0.00004$^{***}$ &  &  &  &  &  &  \\
\hspace*{3mm} in million USD & (0.00001) & (0.00001) &  &  &  &  &  &  \\
 & & & & & & & & \\
Aid flows for transport projects &  &  & $-$0.0001$^{**}$ & $-$0.0001$^{**}$ &  &  &  &  \\
\hspace*{3mm} in million USD &  &  & (0.00002) & (0.00002) &  &  &  &  \\
 & & & & & & & & \\
Total number of aid projects &  &  &  &  & $-$0.002$^{***}$ & $-$0.003$^{***}$ &  &  \\
 &  &  &  &  & (0.0004) & (0.001) &  &  \\
 & & & & & & & & \\
Number of transport projects &  &  &  &  &  &  & $-$0.005$^{***}$ & $-$0.006$^{***}$ \\
 &  &  &  &  &  &  & (0.002) & (0.002) \\
 & & & & & & & & \\
Country FE & Yes & Yes & Yes & Yes & Yes & Yes & Yes & Yes \\
Geographic controls & Yes & Yes & Yes & Yes & Yes & Yes & Yes & Yes \\
Simulation controls &  & Yes &  & Yes &  & Yes &  & Yes \\
Observations & 10,158 & 10,158 & 10,158 & 10,158 & 10,158 & 10,158 & 10,158 & 10,158 \\
R$^{2}$ & 0.124 & 0.126 & 0.123 & 0.125 & 0.125 & 0.128 & 0.124 & 0.126 \\
 \hline
 \hline \\[-1.8ex]
 \textit{Note:}  & \multicolumn{8}{r}{$^{*}$p$<$0.1; $^{**}$p$<$0.05; $^{***}$p$<$0.01} \\
 \end{tabular}


}

\mysubcaption{\lipsum[1-1]}
\end{table}
% End Figure


%-------------------------------------------------

\newpage
\begin{spacing}{1.0}
\setlength{\bibsep}{2.5pt plus 1.5ex}
\bibliography{Thesis_library}
\end{spacing}

\end{document}

%
% % Figure old stuff I chose not to report
% \begin{table}[] \centering
%   \caption{Null Effect of ethnic power relations}
%   \label{tab:Ethn_regs}
%   \resizebox{\textwidth}{!}{
%
%
%   \begin{tabular}{@{\extracolsep{5pt}}lcccccccc}
%   \\[-1.8ex]\hline
%   \hline \\[-1.8ex]
%    & \multicolumn{8}{c}{\textit{Dependent variable: $\Lambda_{i}$}} \\
%    \cline{2-9}
%      \\[-1.8ex]
%    & \multicolumn{2}{c}{\small Grid Cells} & \multicolumn{6}{c}{\small Ethnic Homelands} \\
%   \cline{2-3} \cline{4-9}
%   \\[-1.8ex] & (1) & (2) & (3) & (4) & (5) & (6) & (7) & (8) \\
%   \hline \\[-1.8ex]
%    Ethnicity discriminated against   & $-$0.002 & &  &  & $-$0.001 & 0.0001 & &  \\
%     \hspace*{3mm} at some point 1960--2010  & (0.005) &  && & (0.006) & (0.006) & & \\
%      & & &    & &    &  & \\
%    Ethnicity holding a power monopoly   &  & $-$0.011    &  & &    &  & \\
%     \hspace*{3mm} at some point 1960--2010  &  & (0.008) &    &  &    &  & \\
%      & & &    & &    &  & \\
%     Ethnicity being involved in an ethnic war  & &  & 0.001 & 0.001 & & & &  \\
%  \hspace*{3mm} at some point 1960--2010 & & &(0.007)&(0.007) & & &  &  \\
%   &  & && & & & & \\
%   Ethnicity excluded from the government  & & & & &&  & $-$0.007 & $-$0.006 \\
%    \hspace*{3mm} at some point 1960--2010 & & & && && (0.005)& (0.005) \\
%   &  & & &&&&& \\
%   \hline \\[-1.8ex]
%   Country FE  & Yes & Yes & Yes & Yes & Yes  & Yes & Yes & Yes \\
%   Geographic controls  & Yes & Yes & Yes & Yes & Yes  & Yes & Yes & Yes \\
%   Simulation controls  & Yes & Yes & Yes & Yes & Yes  & Yes & Yes & Yes\\
%   Ethnicity controls  &  &  & & Yes & & Yes & & Yes \\
%   Observations  & 4,401 & 4,401 & 483 & 483 & 483 & 483 & 483 & 483 \\
%   R$^{2}$  & 0.115 & 0.115 & 0.170& 0.181 & 0.170 & 0.181 & 0.173 & 0.183 \\
%   \hline
%   \hline \\[-1.8ex]
%   \textit{Note:}  & \multicolumn{8}{r}{$^{*}$p$<$0.1; $^{**}$p$<$0.05; $^{***}$p$<$0.01} \\
%   \end{tabular}
%
% }
%
% \mysubcaption{This table displays statistically insignificant effects of various indicators of ethnic power relations on the local infrastructure discrimination index. The sample comprises regions of politically relevant ethnic groups. Columns (1) -- (2) conduct regressions on the grid cell level. Column (1) first tests for an association between $\Lambda_{i}$ and an indicator for whether the cell is in an ethnic homeland that was politically discriminated against at some point in 1960--2010 as reported by \cite{Vogt_IntegratingDataEthnicity_2015}. Column (2) similarly reports effects of whether the cell is in an ethnic homeland that held a power monopoly at some point 1960--2010. Columns (3) -- (8) report estimates from ethnicity-level regressions. (3) and (4) analyse the effect of being involved in a major or minor civil conflict between 1960--2010. (5) and (6) analyse ethnic discrimination as in (1). (7) and (8) investigate effects of being excluded from the government at some point between 1960--2010. Geographic controls consist of altitude, temperature, average land suitability, malaria prevalence, yearly growing days, average precipitation, the fourth-order polynomial of latitude and longitude, and an indicator of whether the grid cell lies on the border of a country's network. Simulation controls are comprised of population, night lights, ruggedness, and a dummy for whether a cell is classified as urban. Ethnicity controls consist of the log area of a homeland and a dummy indicating whether the ethnic homeland was split by the arbitrary border drawings from the colonial powers. Ethnicity-level data from \cite{michalopoulos_long-run_2016}. Heteroskedasticity-robust standard errors are clustered on the 3x3 degree level (for columns 1--2) and the country level (columns 3--8) respectively.}
% \end{table}
% % End Figure
%
